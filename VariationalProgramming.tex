\chapter{Variational Programming}
\label{ch:vp}

In this chapter, we establish sound map\footnote{The map operation is part of the definition of a functor.} and bind\footnote{The bind operation is part of the definition of a monad.} operations for option-lists.
We also provide a general definition of soundness for operations on variational data structures.
Variational lists are functional data structures for representation and computation with variation in lists using the choice calculus~\citep{WE12gpce}.
Option-lists are an efficient implementation of variational lists which support sharing of list elements among variant lists~\citep{Walk14onward}.
Variational data structures are described in \secref{var} and variational lists are described in \secref{olist}.
A sound map operation is established in \secref{map} and a sound bind operation is established in \secref{bind}.

\section{Variational Data Structures}
\label{sec:var}

%Throughout this chapter, let \Val\ and \ValY\ be arbitrary types, i.e., type variables.
A \vocab{variational value} of type \Val\ is a value of type \Var[\Val], which is the choice calculus instantiated with type \Val.
The abstract syntax of variational values is described for an arbitrary type \Val\ by the grammar in \figref{varsyn}.
Note that formula choices are used.
However, formula choices could be replaced by dimension choices---along with appropriate changes to selection---and the results of this chapter would still hold.

\begin{figure}[H]
  \onehalfspacing
  \begin{syntax}
    \val \in \Val &&& \textit{Value} \\
    \var \in \Var[\Val]
    & ::= & \val & \\
    & | & \chc{\var, \var} & \textit{Choice}
  \end{syntax}
  \caption{Variational value syntax.}
  \label{fig:varsyn}
\end{figure}

The selection operation for variational values is given by the function \sel, which is defined for an arbitrary type \Val\ by the equations in \figref{vsel}.
In the last equation, note that if $\fSem[\cE]{\fE} \ne \tagL$, then $\fSem[\cE]{\fE} = \tagR$.
This function selects a variant from a variational value.

\begin{figure}[H]
  \onehalfspacing
  \begin{alignat*}{2}
    &\mathrlap{\sel : \cS \to \Var[\Val] \to \Val} \\
    &\sel\ \cE\ \val &&= \val \\
    &\sel\ \cE\ \chc{\var_1, \var_2} &&=
    \begin{cases}
      \sel\ \cE\ \var_1, & \text{if }\fSem[\cE]{\fE} = \tagL \\
      \sel\ \cE\ \var_2, & \text{otherwise}%(\fSem[\cE]{\fE} = \tagR)
    \end{cases}
  \end{alignat*}
  \caption{Selection operation for variational values.}
  \label{fig:vsel}
\end{figure}

For example, consider variational integers.
A variational integer is a variational value with integer variants.
Suppose $\dE_1$ and $\dE_2$ are dimensions.
Then $\chc[\dE_1]{1,\chc[\dE_2]{2,3}}$ is a variational integer.
This variational integer consists of a choice.
The label of the choice is the dimension $\dE_1$ and the alternatives are the integer $1$ and the nested choice $\chc[\dE_2]{2,3}$.
The label of the nested choice is the dimension $\dE_2$ and the alternatives are the integers $2$ and $3$.
Moreover, for each configuration \cE, the image of this variational integer under the partially applied function $\sel\ \cE$ is the following:
%
\begin{inparaenum}[(1)]
  \item if $\cE\ \dE_1 = \tagL$, then the image is the integer $1$,
  \item if $\cE\ \dE_1 = \tagR$ and $\cE\ \dE_2 = \tagL$, then the image is the integer $2$, and
  \item if $\cE\ \dE_1 = \tagR$ and $\cE\ \dE_2 = \tagR$, then the image is the integer $3$.
\end{inparaenum}

%A \vocab{variational function} is a function with type $\Var[\Val] \to \Var[\ValY]$, for some types \Val\ and \ValY.
A function $\varphi : \Var[\Val] \to \Var[\ValY]$ is \vocab{sound} at a configuration $\cE \in \cS$ and with respect to a function $\psi : \Val \to \ValY$ if the equality
$
\sel\ \cE \comp \varphi = \psi \comp \sel\ \cE
$
holds.
Equivalently, the following diagram commutes:
$$
  \begin{tikzcd}
    \Var[\Val] \ar[d, "\sel\ \cE"'] \ar[r, "\varphi"] &
    \Var[\ValY] \ar[d, "\sel\ \cE"] \\
    \Val \ar[r, "\psi"'] & \ValY
  \end{tikzcd}
$$
Moreover, a function $\varphi$ is \vocab{sound} with respect to a function $\psi$ if $\varphi$ is sound at every configuration with respect to $\psi$.
%Whenever the definition of soundness at a configuration is extended to other functions, the previous definition is extended as well.
Informally, a (variational) function is sound at a configuration and with respect to a (plain) function if for each element in the domain of the variational function,
%
\begin{inparaenum}[(1)]
  \item the image under the plain function of the variant selected from the element is equivalent to
  \item the variant selected from the image of the element under the variational function.
\end{inparaenum}

In general, a data structure is given by a type constructor \TConsU\ and the corresponding \vocab{variational data structure} is \TConsVU.
A data structure \TConsW\ \vocab{implements} a variational data structure \TConsVU\ with a selection operation given by a polymorphic function $\wsel : \cS \to \TConsWX \to \TConsUX$ with type variable \Val.

Suppose \TConsW\ implements \TConsVU\ with selection given by \wsel.
Then a function $\varphi : \TConsWX \to \TConsWY$ is \vocab{sound} at a configuration $\cE \in \cS$ and with respect to a function $\psi : \TConsUX \to \TConsUY$ if the equality $\wsel\ \cE \comp \varphi = \psi \comp \wsel\ \cE$ holds.
Equivalently, the following diagram commutes:
$$
  \begin{tikzcd}
    \TConsWX \ar[d, "\wsel\ \cE"'] \ar[r, "\varphi"] &
    \TConsWY \ar[d, "\wsel\ \cE"] \\
    \TConsUX \ar[r, "\psi"'] & \TConsUY
  \end{tikzcd}
$$
Moreover, a function $\varphi$ is \vocab{sound} with respect to a function $\psi$ if $\varphi$ is sound at every configuration with respect to $\psi$.

\section{Variational Lists}
\label{sec:olist}

An \vocab{option} (or \vocab{optional value}) of type \Val\ is a value of the lifted type $\LVal = \Val \cup \set{\Bot}$ where $\Bot \notin \Val$.
The symbol \Bot\ is called the \vocab{bottom value}.
Note that we use the symbol \Bot\ to denote the bottom values of both \LVal\ and \LValY.
A \vocab{variational option} of type \Val\ is a value of type \Var.

A \vocab{list} of elements of type \Val\ is a list of type \Lst.
The standard abstract syntax for lists is described for an arbitrary type \Val\ by the grammar in \figref{listsyn}.
A \vocab{variational list} of elements of type \Val\ is a value of type \Var[(\Lst)].
An \vocab{option-list} of elements of type \Val\ is a list of type \Lst[(\Var)].
For convenience, we write this as \OLst, i.e., $\OLst = \Lst[(\Var)]$.
%The syntax for option-lists is described by the grammar in \figref{olistsyn}.

\begin{figure}[H]
  \onehalfspacing
  \begin{syntax}
    \val \in \Val &&& \textit{Element} \\
    \lst \in \Lst
    & ::= & \nil & \textit{Nil} \\
    & | & \val \cons \lst & \textit{Cons}
  \end{syntax}
  \caption{List syntax.}
  \label{fig:listsyn}
\end{figure}

%\begin{figure}[H]
%  \onehalfspacing
%  \begin{syntax}
%    \lst \in \OLst
%    & ::= & \nil & \textit{Nil} \\
%    & | & \var \cons \lst & \textit{Cons}
%  \end{syntax}
%  \caption{Option-list syntax.}
%  \label{fig:olistsyn}
%\end{figure}

Option-lists are an implementation of variational lists with selection given by the function \osel, which is defined for an arbitrary type \Val\ by the equations in \figref{osel}.
In the last equation, note that if $\sel\ \cE\ \var \ne \val$, then $\sel\ \cE\ \var = \bot$ since $\var \in \Var$ and so $\sel\ \cE\ \var \in \LVal$.
This function selects a variant list from an option-list by selecting variant elements from the option-list elements and discarding those which are the bottom value.

\begin{figure}[H]
  \onehalfspacing
  \begin{alignat*}{2}
    &\mathrlap{\osel : \cS \to \OLst \to \Lst} \\
    &\osel\ \cE\ \nil &&= \nil \\
    &\osel\ \cE\ (\var \cons \lst) &&=
    \begin{cases}
      \val \cons \osel\ \cE\ \lst, & \text{if }\sel\ \cE\ \var = \val \\
      \osel\ \cE\ \lst, & \text{otherwise}%(\sel\ \cE\ \var = \bot)
    \end{cases}
  \end{alignat*}
  \caption{Selection operation for option-list.}
  \label{fig:osel}
\end{figure}

%A function $\varphi : \OLst \to \OLst[\ValY]$ is \vocab{sound} at a configuration $\cE \in \cS$ and with respect to a function $\psi : \Lst \to \Lst[\ValY]$ if the equality $\osel\ \cE \comp \varphi = \psi \comp \osel\ \cE$ holds.
%Equivalently, the following diagram commutes:
%$$
%  \begin{tikzcd}
%    \OLst \ar[d, "\osel\ \cE"'] \ar[r, "\varphi"] &
%    \OLst[\ValY] \ar[d, "\osel\ \cE"] \\
%    \Lst \ar[r, "\psi"'] & \Lst[\ValY]
%  \end{tikzcd}
%$$
%$$
%  \begin{tikzcd}[row sep=scriptsize, column sep=scriptsize]
%    & 1 \ar[dl] \ar[rr] \ar[dd] & & 2 \ar[dl] \ar[dd] \\
%    3 \ar[rr, crossing over] \ar[dd] & & 4 \\
%    & 5 \ar[dl] \ar[rr] &  & 6 \ar[dl] \\
%    7 \ar[rr] & & 8 \ar[from=uu, crossing over]\\
%  \end{tikzcd}
%$$

For example, consider lists and option-lists of integers.
For convenience, we write the concrete syntax of lists as comma separated sequences of elements delimited by square brackets, e.g., \clst{1, 2, 3} and \clst{2, 4} are lists of integers in the concrete syntax which correspond to the lists of integers $1 \cons 2 \cons 3 \cons \nil$ and $2 \cons 4 \cons \nil$, respectively, in the abstract syntax.
%A list of integers is a list where the elements are integers, e.g.,  $\lbrack 1, 2, 3\rbrack$ and $\lbrack 2, 4\rbrack$ are lists of integers.
%Note that we omit the last cons and nil constructs when writing specific examples of lists.
An option list of integers is a list where the elements are variational options and the non-bottom variants are integers.
Suppose \dE\ is a dimension.
Then \clst{\chc[\dE]{1, \bot}, 2, \chc[\dE]{3, 4}} is an option-list of integers.
%The first element of this list is the choice $\chc[\dE]{1, \bot}$, with label \dE\ and alternatives $1$ and $\bot$.
%The second element of this list is the integer $2$.
%The third element of this list is the choice $\chc[\dE]{3, 4}$, with label \dE\ and alternatives $3$ and $4$.
Moreover, for each configuration \cE, the image of this option-list under the partially applied function $\osel\ \cE$ is the following:
%
\begin{inparaenum}[(1)]
  \item if $\cE\ \dE = \tagL$, then the image is the variant list \clst{1, 2, 3} and
  \item if $\cE\ \dE = \tagR$, then the image is the variant list \clst{2, 4}.
\end{inparaenum}

\subsection{Map Operation}
\label{sec:map}

The list type constructor \Lst[(\cdot)] together with some map operation form a functor~\citep{Wad92fp,Wad92monad,Wad95monad}.
The standard map operation for lists is given by the function \lmap, which is defined for arbitrary types \Val\ and \ValY\ by the equations in \figref{map}.
This function maps a function over the elements of a list.

\begin{figure}[H]
  \onehalfspacing
  \begin{alignat*}{2}
    &\mathrlap{\lmap : (\Val \to \ValY) \to \Lst \to \Lst[\ValY]} \\
    &\lmap\ \varphi\ \nil &&= \nil \\
    &\lmap\ \varphi\ (\val \cons \lst)
    &&= \varphi\ \val \cons \lmap\ \varphi\ \lst
  \end{alignat*}
  \caption{Map operation for list.}
  \label{fig:map}
\end{figure}

%For example, consider applying the function \lmap\ to the function \lam{x}{x^2} and the list of integers $1 \cons 2 \cons 3$.
%The image of the list of integers $1 \cons 2 \cons 3$ under the partially applied function $\lmap\ (\lam{x}{x^2})$ is the list of integers $1 \cons 4 \cons 9$.

It can be shown that the option-list type constructor \OLst[(\cdot)] together with some map operation form a functor.
The map operation for option-lists is given by the function \omap, which is defined for arbitrary types \Val\ and \ValY\ by the equations in \figref{omap}.
This function maps a function over the elements of an option-list.

\begin{figure}[H]
  \onehalfspacing
  \begin{alignat*}{2}
    &\mathrlap{\omap : (\Val \to \ValY) \to \OLst \to \OLst[\ValY]} \\
    &\omap\ \varphi\ \nil &&= \nil \\
    &\omap\ \varphi\ (\var \cons \lst)
    &&= \hmap\ \varphi\ \var \cons \omap\ \varphi\ \lst
  \end{alignat*}
  \caption{Map operation for option-lists.}
  \label{fig:omap}
\end{figure}

The function \hmap\ is defined for arbitrary types \Val\ and \ValY\ by the equations in \figref{vmap}.
This function maps a function over the variants of a variational option.

\begin{figure}[H]
  \onehalfspacing
  \begin{alignat*}{2}
    &\mathrlap{\hmap : (\Val \to \ValY) \to \Var \to \VarY} \\
    &\hmap\ \varphi\ \Bot &&= \Bot \\
    &\hmap\ \varphi\ \val &&= \varphi\ \val \\
    &\hmap\ \varphi\ \chc{\var_1, \var_2}
    &&= \chc{\hmap\ \varphi\ \var_1, \hmap\ \varphi\ \var_2}
  \end{alignat*}
  \caption{\hmap\ function definition.}
  \label{fig:vmap}
\end{figure}

%For example, consider applying the function \omap\ to the function \lam{x}{x^2} and the option-list of integers $\chc[\dE]{1, 2} \cons 3 \cons \chc[\dE]{4, \bot}$.
%The image of the option-list of integers $\chc[\dE]{1, 2} \cons 3 \cons \chc[\dE]{4, \bot}$ under the partially applied function $\omap\ (\lam{x}{x^2})$ is the option-list of integers $\chc[\dE]{1, 4} \cons 9 \cons \chc[\dE]{16, \bot}$.

We use \lemref{vselnone} and \lemref{vselsome} to prove soundness of the map operation for option-lists.
Informally, the first lemma says that mapping a function over a variant which is the bottom value does not change the value of the variant and the second lemma says that mapping a function over a variant which is a (non-bottom) value results in application of the function to that value.

\begin{lemma}
  \label{lem:vselnone}
  For all configurations $\cE \in \cS$, functions $\varphi : \Val \to \ValY$, and variational options $\var \in \Var$, if
  $
    \sel\ \cE\ \var = \Bot
  $,
  then
  $
    \sel\ \cE\ (\hmap\ \varphi\ \var) = \Bot
  $.
\end{lemma}

\begin{proof}
  We will use structural induction on $\var$.

  For the first base case, suppose $\var = \Bot$.
  Then by the definitions of \hmap\ and \sel\ we have $\sel\ \cE\ (\hmap\ \varphi\ \Bot) = \Bot$ and we are done.
  For the second base case, suppose $\var = \val$ for some value $\val \in \Val$.
  Then by the definition of \sel\ and the hypothesis of the lemma we have $\val = \sel\ \cE\ \val = \Bot$, which is a contradiction and so this case is impossible.

  For the induction step, suppose $\var = \chc{\var_1, \var_2}$ for some formula $\fE \in \fS$ and variational options $\var_1, \var_2 \in \Var$ with the following inductive hypotheses:
  \begin{align*}
    \sel\ \cE\ \var_1 = \Bot &\implies
    \sel\ \cE\ (\hmap\ \varphi\ \var_1) = \Bot \\
    \sel\ \cE\ \var_2 = \Bot &\implies
    \sel\ \cE\ (\hmap\ \varphi\ \var_2) = \Bot
  \end{align*}
  Notice that \fSem[\cE]{\fE} is either \tagL\ or \tagR.

  If $\fSem[\cE]{\fE} = \tagL$, then by the definition of \sel\ and the hypothesis of the lemma we have $\sel\ \cE\ \var_1 = \sel\ \cE\ \chc{\var_1, \var_2} = \Bot$.
  By applying this result to our first inductive hypothesis, we have the following derivation by the definitions of \hmap\ and \sel.
  \begin{align*}
    \sel\ \cE\ (\hmap\ \varphi\ \chc{\var_1, \var_2})
    &= \sel\ \cE\ \chc{\hmap\ \varphi\ \var_1, \hmap\ \varphi\ \var_2} \\
    &= \sel\ \cE\ (\hmap\ \varphi\ \var_1) \\
    &= \Bot
  \end{align*}
  The case where $\fSem[\cE]{\fE} = \tagR$ follows by similar reasoning, see \appref{vp-coq}.
\end{proof}

\begin{lemma}
  \label{lem:vselsome}
  For all configurations $\cE \in \cS$, functions $\varphi : \Val \to \ValY$, and variational options $\var \in \Var$, if
  $
    \sel\ \cE\ \var = \val
  $
  for some value $\val\in\Val$, then
  $
    \sel\ \cE\ (\hmap\ \varphi\ \var) = \varphi\ \val
  $.
\end{lemma}

\begin{proof}
  We will use structural induction on $\var$.

  For the first base case, suppose $\var = \Bot$.
  Then by the definition of \sel\ and the hypothesis of the lemma we have $\Bot = \sel\ \cE\ \Bot = \val$, which is a contradiction and so this case is impossible.
  For the second base case, suppose $\var = \val'$ for some value $\val' \in \Val$.
  Then by the definition of \sel\ and the hypothesis of the lemma we have $\val' = \sel\ \cE\ \val' = \val$.
  By the definitions of \hmap\ and \sel, this means
  $
    \sel\ \cE\ (\hmap\ \varphi\ \val')
    = \varphi\ \val'
    = \varphi\ \val
  $.

  For the induction step, suppose $\var = \chc{\var_1, \var_2}$ for some formula $\fE \in \fS$ and variational options $\var_1, \var_2 \in \Var$ with the following inductive hypotheses:
  \begin{align*}
    \sel\ \cE\ \var_1 = \val &\implies
    \sel\ \cE\ (\hmap\ \varphi\ \var_1) = \val \\
    \sel\ \cE\ \var_2 = \val &\implies
    \sel\ \cE\ (\hmap\ \varphi\ \var_2) = \val
  \end{align*}
  Notice that \fSem[\cE]{\fE} is either \tagL\ or \tagR.

  If $\fSem[\cE]{\fE} = \tagL$, then by the definition of \sel\ and the hypothesis of the lemma we have $\sel\ \cE\ \var_1 = \sel\ \cE\ \chc{\var_1, \var_2} = \val$.
  By applying this result to our first inductive hypothesis, we have the following derivation by the definitions of \hmap\ and \sel.
  \begin{align*}
    \sel\ \cE\ (\hmap\ \varphi\ \chc{\var_1, \var_2})
    &= \sel\ \cE\ \chc{\hmap\ \varphi\ \var_1, \hmap\ \varphi\ \var_2} \\
    &= \sel\ \cE\ (\hmap\ \varphi\ \var_1) \\
    &= \val
  \end{align*}
  The case where $\fSem[\cE]{\fE} = \tagR$ follows by similar reasoning, see \appref{vp-coq}.
\end{proof}

We use the definition of soundness for option-lists to define soundness of the map operation for option-lists.
The map operation for option-lists is \vocab{sound} if for each function $\varphi : \Val \to \ValY$, the function $\omap\ \varphi$ is sound with respect to the function $\lmap\ \varphi$.
This is stated formally by \thmref{map}.
A machine verified version of the proof of this theorem is presented in \appref{vp-coq}.

\begin{theorem}
  \label{thm:map}
  For all configurations $\cE \in \cS$ and functions $\varphi : \Val \to \ValY$, the equality
  $
    \osel\ \cE \comp \omap\ \varphi =
    \lmap\ \varphi \comp \osel\ \cE
  $
  holds.
  Equivalently, the following diagram commutes:
  $$
    \begin{tikzcd}
      \OLst \ar[d, "\osel\ \cE"'] \ar[r, "\omap\ \varphi"] &
      \OLst[\ValY] \ar[d, "\osel\ \cE"] \\
      \Lst \ar[r, "\lmap\ \varphi"'] &
      \Lst[\ValY]
    \end{tikzcd}
  $$
\end{theorem}

\begin{proof}
  We will show that the equality
  $
    (\osel\ \cE \comp \omap\ \varphi)\ \lst =
    (\lmap\ \varphi \comp \osel\ \cE)\ \lst
  $
  holds for all option-lists $\lst \in \OLst$ by structural induction on $\lst$.

  For the base case, suppose that $\lst = \nil$.
  Then we have the following derivation by the definitions of \osel, \lmap, and \omap.
  \begin{align*}
    (\osel\ \cE \comp \omap\ \varphi)\ \nil
    &= \osel\ \cE\ \nil \\
    &= \nil \\
    &= \lmap\ \varphi\ \nil \\
    &= (\lmap\ \varphi \comp \osel\ \cE)\ \nil
  \end{align*}

  For the induction step, suppose that $\lst = \var \cons \lst'$, for some variational option $\var \in \Var$ and option-list $\lst' \in \OLst$ with
  $
    (\osel\ \cE \comp \omap\ \varphi)\ \lst' =
    (\lmap\ \varphi \comp \osel\ \cE)\ \lst'
  $.
  Notice that $\sel\ \cE\ \var$ is either \Bot\ or some value $\val\in\Val$.

  If $\sel\ \cE\ \var = \Bot$, then $\sel\ \cE\ (\hmap\ \varphi\ \var) = \Bot$ by \lemref{vselnone} and we have the following derivation by the definitions of \osel\ and \omap\ and our inductive hypothesis.
  \begin{align*}
    (\osel\ \cE \comp \omap\ \varphi)\ (\var \cons \lst')
    &= \osel\ \cE\ (\hmap\ \varphi\ \var \cons \omap\ \varphi\ \lst') \\
    &= \osel\ \cE\ (\omap\ \varphi\ \lst') \\
    &= (\lmap\ \varphi \comp \osel\ \cE)\ \lst' \\
    &= (\lmap\ \varphi \comp \osel\ \cE)\ (\var \cons \lst')
  \end{align*}

  If $\sel\ \cE\ \var = \val$ for some value $\val \in \Val$, then $\sel\ \cE\ (\hmap\ \varphi\ \var) = \varphi\ \val$ by \lemref{vselsome} and we have the following derivation by the definitions of \osel, \lmap, and \omap, and our inductive hypothesis.
  \begin{align*}
    (\osel\ \cE \comp \omap\ \varphi)\ (\var \cons \lst')
    &= \osel\ \cE\ (\hmap\ \varphi\ \var \cons \omap\ \varphi\ \lst') \\
    &= \varphi\ \val \cons \osel\ \cE\ (\omap\ \varphi\ \lst') \\
    &= \varphi\ \val \cons (\lmap\ \varphi \comp \osel\ \cE)\ \lst' \\
    &= \lmap\ \varphi\ (\val \cons \osel\ \cE\ \lst') \\
    &= (\lmap\ \varphi \comp \osel\ \cE)\ (\var \cons \lst')
  \end{align*}
  This completes the proof.
\end{proof}

\subsection{Bind Operation}
\label{sec:bind}

The list type constructor \Lst[(\cdot)] together with some bind and unit operations form a monad on the category of types~\citep{Wad92fp,Wad92monad,Wad95monad}.
The standard bind operation for lists is given by the function \lbind, which is defined for arbitrary types \Val\ and \ValY\ by the equations in \figref{bind}.
This function applies a function to the elements of a list and joins the results.
The infix binary operator ($\concat$) denotes a concatenation (or append) operation for lists.
The bind operation for lists is sometimes referred to as a concatenation-map (or flat-map) operation in programming languages.

\begin{figure}[H]
  \onehalfspacing
  \begin{alignat*}{2}
    &\mathrlap{\lbind : (\Val \to \Lst[\ValY]) \to \Lst \to \Lst[\ValY]} \\
    &\lbind\ \varphi\ \nil &&= \nil \\
    &\lbind\ \varphi\ (\val\cons\lst)
    &&= \varphi\ \val \concat \lbind\ \varphi\ \lst
  \end{alignat*}
  \caption{Bind operation for lists.}
  \label{fig:bind}
\end{figure}

It can be shown that the option-list type constructor \OLst[(\cdot)] together with some bind and unit operations form a monad on the category of types.
The bind operation for option-lists is given by the function \obind, which is defined for arbitrary types \Val\ and \ValY\ by the equations in \figref{obind}.
This function applies a function to the elements of an option-list and joins the results.

\begin{figure}[H] % htb
  \onehalfspacing
  \begin{alignat*}{2}
    &\mathrlap{\obind : (\Val \to \OLst[\ValY]) \to \OLst \to \OLst[\ValY]} \\
    &\obind\ \varphi\ \nil &&= \nil \\
    &\obind\ \varphi\ (\var\cons\lst)
    &&= \hbind\ \varphi\ \var \concat \obind\ \varphi\ \lst
  \end{alignat*}
  \caption{Bind operation for option-lists.}
  \label{fig:obind}
\end{figure}

The function \hbind\ is defined for arbitrary types \Val\ and \ValY\ by the equations in \figref{hbind}.
This function applies a function to the variants of a variational option and joins the results.

\begin{figure}[H]
  \onehalfspacing
  \begin{alignat*}{2}
    &\mathrlap{\hbind : (\Val \to \OLst[\ValY]) \to \Var \to \OLst[\ValY]} \\
    &\hbind\ \varphi\ \Bot &&= \nil \\
    &\hbind\ \varphi\ \val &&= \varphi\ \val \\
    &\hbind\ \varphi\ \chc{\var_1, \var_2}
    &&= \hzip\ \fE\ (\hbind\ \varphi\ \var_1)\ (\hbind\ \varphi\ \var_2)
  \end{alignat*}
  \caption{\hbind\ function definition.}
  \label{fig:hbind}
\end{figure}

The function \hzip\ is defined for an arbitrary type \Val\ by the equations in \figref{ozip}.
This function forms an option-list where the elements are choices with a given formula as a label and elements of two given option-lists as alternatives.

\begin{figure}[H]
  \onehalfspacing
  %\begin{alignat*}{3}
  %  &\mathrlap{\hzip : \fS \to \OLst \to \OLst \to \OLst} \\
  %  &\hzip\ \fE\ \nil\ \nil &&= \nil \\
  %  &\hzip\ \fE\ \nil\ (\var\cons\lst)
  %  &&= \chc{\Bot, \var}
  %  &&\cons\hzip\ \fE\ \nil\ \lst \\
  %  &\hzip\ \fE\ (\var\cons\lst)\ \nil
  %  &&= \chc{\var, \Bot}
  %  &&\cons\hzip\ \fE\ \lst\ \nil \\
  %  &\hzip\ \fE\ (\var_1\cons\lst_1)\ (\var_2\cons\lst_2)
  %  &&= \chc{\var_1, \var_2}
  %  &&\cons\hzip\ \fE\ \lst_1\ \lst_2
  %\end{alignat*}
  \begin{alignat*}{2}
    &\mathrlap{\hzip : \fS \to \OLst \to \OLst \to \OLst} \\
    &\hzip\ \fE\ \nil\ \lst
    &&= \lmap\ (\lam{\var}{\chc{\Bot, \var}})\ \lst \\
    &\hzip\ \fE\ \lst\ \nil
    &&= \lmap\ (\lam{\var}{\chc{\var, \Bot}})\ \lst \\
    &\hzip\ \fE\ (\var_1\cons\lst_1)\ (\var_2\cons\lst_2)
    &&= \chc{\var_1, \var_2} \cons\hzip\ \fE\ \lst_1\ \lst_2
  \end{alignat*}
  \caption{\hzip\ function definition.}
  \label{fig:ozip}
\end{figure}

We use the definition of soundness (at a configuration) to define soundness of the bind operation for option-lists.
The bind operation for option-lists is \vocab{sound} if for each configuration $\cE \in \cS$ and function $\varphi : \Val \to \OLst[\ValY]$, the function $\obind\ \varphi$ is sound at \cE\ and with respect to the function $\lbind\ \psi$, where $\psi = \osel\ \cE \comp \varphi$.
This is stated formally by \thmref{bind}.
We prove this theorem by structural induction on option-lists but the proof is rather tedious and so it is omitted here, see \appref{vp-coq}.

\begin{theorem}
  \label{thm:bind}
  For all configurations $\cE \in \cS$ and functions $\varphi : \Val \to \OLst$, the equality
  $
    \osel\ \cE \comp \obind\ \varphi = \lbind\ \psi \comp \osel\ \cE
  $
  holds, where $\psi = \osel\ \cE \comp \varphi$.
  Equivalently, the following diagram commutes:
  $$
    \begin{tikzcd}
      \OLst \ar[d, "\osel\ \cE"'] \ar[r, "\obind\ \varphi"] &
      \OLst[\ValY] \ar[d, "\osel\ \cE"] \\
      \Lst \ar[r, "\lbind\ \psi"'] &
      \Lst[\ValY]
    \end{tikzcd}
  $$
\end{theorem}
