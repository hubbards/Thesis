\chapter{Introduction}
\label{ch:intro}

In this thesis, we present semantic equivalence rules for an extension of the choice calculus and sound operations for an implementation of variational lists.
The choice calculus is a calculus for describing variation and the formula choice calculus is an extension with formulas.
We prove semantic equivalence rules for the formula choice calculus.
Variational lists are functional data structures for representing and computing with variation in lists using the choice calculus.
We prove map and bind operations are sound for an implementation of variational lists.
These proofs are written and verified in the language of the Coq proof assistant.

The choice calculus is a metalanguage for describing variation in an arbitrary object language~\citep{EW11tosem}.
The formula choice calculus is an extension of the choice calculus where choices are labeled with formulas instead of dimensions~\citep{WO14gpce}.
Semantic equivalence rules for the choice calculus have been established by previous work~\citep{Walk13thesis}.
In \chref{cc}, we establish similar semantic equivalence rules for the formula choice calculus.

Previous work has used the formula choice calculus without proving semantic equivalence rules.
For example, a projectional editing model of variational software is based on formula choice calculus and some semantic equivalence rules have been stated---without proof---for this projectional editing model~\citep{WO14gpce}.
As another example, TypeChef is a tool for parsing and type checking C code with preprocessor directives and is based on a model similar to the formula choice calculus~\citep{Ken10,Kas11,Walk14onward}.
We provide a formal foundation for previous and future work that use the formula choice calculus.

Variational lists are functional data structures for representing and computing with variation in lists using the choice calculus~\citep{WE12gpce}.
Option-lists are an implementation of variational lists as lists of variational optional values~\citep{Walk14onward}.
%A (covariant) endofunctor on the category of types is a type constructor together with a map operation which satisfies certain properties.
%A monad on the category of types is a type constructor together with some bind and unit operations which satisfy certain properties.
In \chref{vp}, we establish sound map\footnote{The map operation is part of the definition of a functor.} and bind\footnote{The bind operation is part of the definition of a monad.} operations for option-lists.
We also provide a general definition of soundness for operations on variational data structures.

Variational lists are common variational data structures that arise naturally in many variational programming problems.
For example, SPLlift is a tool for inter-procedural data-flow analysis on software product lines that uses variational graphs~\citep{Bod13}.
A variational list could be used for an adjacency list representation of a variational graph~\citep{Walk14onward}.
As another example, the CIDE tool~\citep{KAK08} and Color Featherweight Java~\citep{Kas12} use data structures which are similar to---but less expressive than---option-lists~\citep{Walk14onward}.
Previous work has identified a need for foundational research on variational data structures~\citep{Walk14onward}.
We provide a formal foundation for option-lists.
We also demonstrate a general and principled technique for establishing soundness for operations on variational data structures.

In \appref{coq}, we provide verified proofs written in the language of the Coq proof assistant~\citep{BC04}.
The properties from \chref{cc} are verified in \appref{f-coq} and \appref{fcc-coq} and the source code is available online.\footnote{\url{https://github.com/hubbards/fcc-coq}}
The properties from \chref{vp} are verified in \appref{vp-coq} and the source code is also available online.\footnote{\url{https://github.com/hubbards/vp-coq}}
