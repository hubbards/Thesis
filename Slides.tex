% slides for final exam

% preamble
% draft,handout,professionalfont,notes=show|only,xcolor=dvipsnames
\documentclass[draft,notes=show,xcolor=dvipsnames]{beamer}

\usepackage[utf8]{inputenc}
\usepackage[T1,OT1]{fontenc}
\usepackage[expansion,protrusion]{microtype}
\usepackage{mathpazo}
\usepackage[scaled]{beramono}
%\usepackage{natbib}
\usepackage{mathpartir}
\usepackage[final]{listings}
\usepackage{tikz-cd}
\usepackage{lambda,cc}

% overlap macros for mathmode
\def\clap#1{\hbox to 0pt{\hss#1\hss}}
\def\mathllap{\mathpalette\mathllapinternal}
\def\mathrlap{\mathpalette\mathrlapinternal}
\def\mathclap{\mathpalette\mathclapinternal}
\def\mathllapinternal#1#2{\llap{$\mathsurround=0pt#1{#2}$}}
\def\mathrlapinternal#1#2{\rlap{$\mathsurround=0pt#1{#2}$}}
\def\mathclapinternal#1#2{\clap{$\mathsurround=0pt#1{#2}$}}

% math operators
\DeclareMathOperator{\Forall}{\forall}
\DeclareMathOperator{\Exists}{\exists}
\DeclareMathOperator{\Not}{\lnot}
% function mapping
\newcommand*{\fdef}[1]{\overset{#1}{\mapsto}}
% placeholder
\newcommand*{\placeholder}{\underline{\hspace{8pt}}}

% language settings for coq
% language definition for coq

\lstdefinelanguage{coq}{
  % case sensitive
  sensitive=true,
  % basic keywords
  morekeywords=[1]{Module,Section,Load,Require,Export,Import,End,Open,Scope,
    Arguments,Global,Instance,Tactic,Notation,Infix,SearchAbout,Eval,Check,
    Inductive,Definition,Fixpoint,Ltac,Fact,Example,Theorem,Lemma,Corollary,
    Fact,Remark,Set,Prop,Type,Proof,Admitted,Abort,Restart,Qed,Defined,forall,
    match,with,by,let,in,as,end,fun,if,then,else},
  % tactics
  morekeywords=[2]{intro,intros,reflexivity,symmetry,transitivity,destruct,
    inversion,induction,simpl,rewrite,apply,assert,unfold,split,left,right,
    omega,try,repeat},
  % comments
  morecomment=[s]{(*}{*)},
  % strings
  morestring=[m]"
}


% lstlisting settings
\lstset{
  language=coq,
  tabsize=2,
  sensitive=true,
  extendedchars=true,
  showstringspaces=false,
  basicstyle=\fontencoding{T1}\ttfamily\scriptsize\color{Black},
  %identifierstyle={},
  %keywordstyle=[1]{},
  %keywordstyle=[2]{},
  commentstyle={\itshape},
  %stringstyle={},
  frame=single,
}

% commutative diagrams
\tikzcdset{row sep/normal=4em,column sep/normal=4em}

% customize inference rules
\renewcommand{\rn}[1]{{\scriptsize\MakeUppercase{#1}}}
% TODO: what does this do?
\renewcommand{\DefTirName}[1]{\rn{#1}}
\def\MathparLineskip{\lineskiplimit=.4em\lineskip=.6em plus 0.2em}

% customize colors / formatting
\renewcommand{\OB}[1]{\ensuremath{#1}}
\renewcommand{\CCkeyw}[1]{\OB{\mathbf{#1}}}
\renewcommand{\CCname}[1]{\OB{\textit{#1}}}
% vocabulary word / definition
\newcommand*{\vocab}[1]{\textbf{\textcolor{magenta}{#1}}}

% rule and language names
% rule names

% separator
\newcommand{\rSep}{\rn{-}}
% constructs
\newcommand*{\rObj}{\rn{Obj}}
\newcommand*{\rChc}{\rn{Chc}}
\newcommand*{\rComp}{\rn{Comp}}
\newcommand*{\rMeet}{\rn{Meet}}
\newcommand*{\rJoin}{\rn{Join}}
% other components
\newcommand*{\rCO}{\rn{CO}}
\newcommand*{\rCC}{\rn{CC}}
\newcommand*{\rSwap}{\rSep\rn{Swap}}
\newcommand*{\rMerge}{\rSep\rn{Merge}}
\newcommand*{\rE}{\rSep\rn{E}}
\newcommand*{\rF}{\rSep\rn{F}}
\newcommand*{\rL}{\rSep\rn{l}}
\newcommand*{\rR}{\rSep\rn{r}}
\newcommand*{\rFL}{\rSep\rn{L}}
\newcommand*{\rFR}{\rSep\rn{R}}
\newcommand*{\rFLL}{\rSep\rn{1}}
\newcommand*{\rFLR}{\rSep\rn{2}}
\newcommand*{\rFRL}{\rSep\rn{3}}
\newcommand*{\rFRR}{\rSep\rn{4}}
\newcommand*{\rRefl}{\rn{Refl}}
\newcommand*{\rSymm}{\rn{Symm}}
\newcommand*{\rTran}{\rn{Tran}}
\newcommand*{\rCong}{\rSep\rn{Cong}}
\newcommand*{\rAssoc}{\rSep\rn{Assoc}}
\newcommand*{\rComm}{\rSep\rn{Comm}}
\newcommand*{\rDist}{\rSep\rn{Dist}}
\newcommand*{\rId}{\rSep\rn{Id}}
\newcommand*{\rAnn}{\rSep\rn{Ann}}
\newcommand*{\rIdemp}{\rSep\rn{Idemp}}
% formula equivalence
\newcommand*{\rReflF}{\rRefl\rF}
\newcommand*{\rSymmF}{\rSymm\rF}
\newcommand*{\rTranF}{\rTran\rF}
% formula congruence
\newcommand*{\rCompCong}{\rComp\rCong}
\newcommand*{\rJoinCong}{\rJoin\rCong}
\newcommand*{\rJoinCongL}{\rJoinCong\rL}
\newcommand*{\rJoinCongR}{\rJoinCong\rR}
\newcommand*{\rMeetCong}{\rMeet\rCong}
\newcommand*{\rMeetCongL}{\rMeetCong\rL}
\newcommand*{\rMeetCongR}{\rMeetCong\rR}
% boolean algebra
\newcommand*{\rCompL}{\rComp\rL}
\newcommand*{\rCompR}{\rComp\rR}
\newcommand*{\rCompIdemp}{\rComp\rIdemp}
\newcommand*{\rCompJoin}{\rComp\rSep\rJoin}
\newcommand*{\rCompMeet}{\rComp\rSep\rMeet}
\newcommand*{\rJoinId}{\rJoin\rId}
\newcommand*{\rJoinIdR}{\rJoinId\rR}
\newcommand*{\rJoinIdL}{\rJoinId\rL}
\newcommand*{\rJoinComp}{\rJoin\rSep\rComp}
\newcommand*{\rJoinCompL}{\rJoinComp\rL}
\newcommand*{\rJoinCompR}{\rJoinComp\rR}
\newcommand*{\rJoinAnn}{\rJoin\rAnn}
\newcommand*{\rJoinAnnL}{\rJoinAnn\rL}
\newcommand*{\rJoinAnnR}{\rJoinAnn\rR}
\newcommand*{\rJoinAssoc}{\rJoin\rAssoc}
\newcommand*{\rJoinComm}{\rJoin\rComm}
\newcommand*{\rJoinDist}{\rJoin\rDist}
\newcommand*{\rJoinIdemp}{\rJoin\rIdemp}
\newcommand*{\rMeetId}{\rMeet\rId}
\newcommand*{\rMeetIdL}{\rMeetId\rL}
\newcommand*{\rMeetIdR}{\rMeetId\rR}
\newcommand*{\rMeetComp}{\rMeet\rSep\rComp}
\newcommand*{\rMeetCompL}{\rMeetComp\rL}
\newcommand*{\rMeetCompR}{\rMeetComp\rR}
\newcommand*{\rMeetAnn}{\rMeet\rAnn}
\newcommand*{\rMeetAnnL}{\rMeetAnn\rL}
\newcommand*{\rMeetAnnR}{\rMeetAnn\rR}
\newcommand*{\rMeetAssoc}{\rMeet\rAssoc}
\newcommand*{\rMeetComm}{\rMeet\rComm}
\newcommand*{\rMeetDist}{\rMeet\rDist}
\newcommand*{\rMeetIdemp}{\rMeet\rIdemp}
% expression equivalence
\newcommand*{\rReflE}{\rRefl\rE}
\newcommand*{\rSymmE}{\rSymm\rE}
\newcommand*{\rTranE}{\rTran\rE}
% choice transposition
\newcommand*{\rCTrans}{\rChc\rSep\rTran}
% choice congruence
\newcommand*{\rOCong}{\rObj\rCong}
\newcommand*{\rOCongL}{\rOCong\rL}
\newcommand*{\rOCongR}{\rOCong\rR}
\newcommand*{\rCCong}{\rChc\rCong}
\newcommand*{\rCCongF}{\rCCong\rF}
\newcommand*{\rCCongL}{\rCCong\rL}
\newcommand*{\rCCongR}{\rCCong\rR}
% choice simplification
\newcommand*{\rCL}{\rChc\rFL}
\newcommand*{\rCR}{\rChc\rFR}
\newcommand*{\rCIdemp}{\rChc\rIdemp}
% choice formula
\newcommand*{\rCJoin}{\rChc\rSep\rJoin}
\newcommand*{\rCJoinC}{\rCJoin\rSep\rComp}
\newcommand*{\rCMeet}{\rChc\rSep\rMeet}
\newcommand*{\rCMeetC}{\rCMeet\rSep\rComp}
% choice distribution
\newcommand*{\rCOSwap}{\rCO\rSwap}
\newcommand*{\rCOSwapL}{\rCOSwap\rL}
\newcommand*{\rCOSwapR}{\rCOSwap\rR}
\newcommand*{\rCCSwap}{\rCC\rSwap}
\newcommand*{\rCCSwapL}{\rCCSwap\rL}
\newcommand*{\rCCSwapR}{\rCCSwap\rR}
% choice merge
\newcommand*{\rCCMerge}{\rCC\rMerge}
\newcommand*{\rCCMergeL}{\rCCMerge\rL}
\newcommand*{\rCCMergeR}{\rCCMerge\rR}

% language names

% tag
\newcommand*{\tagE}{\OB{t}}
\newcommand*{\tagS}{\OB{T}}
\newcommand*{\tagL}{\CCkeyw{L}}
\newcommand*{\tagR}{\CCkeyw{R}}
\newcommand*{\tagLL}{\CCkeyw{1}}
\newcommand*{\tagLR}{\CCkeyw{2}}
\newcommand*{\tagRL}{\CCkeyw{3}}
\newcommand*{\tagRR}{\CCkeyw{4}}
% dimension
\newcommand*{\dE}{\OB{d}}
\newcommand*{\dS}{\OB{D}}
% configuration
\newcommand*{\cE}{\OB{c}}
\newcommand*{\cS}{\OB{C}}
% formula
\newcommand*{\fE}{\OB{f}}
\newcommand*{\fS}{\OB{F}}
\newcommand*{\fSem}[2][{}]{\OB{\fS\sem[#1]{#2}}}
% object language
\newcommand*{\aE}{\OB{a}}
\newcommand*{\aS}{\OB{A}}
\newcommand*{\objE}{\OB{x}}
\newcommand*{\objS}{\OB{X}}
\newcommand*{\objEmpty}{\OB{\varepsilon}}
\renewcommand*{\tr}[2][\aE]{\OB{#1\sub{#2}}} % TODO: rewrite this
% choice calculus
\newcommand*{\ccE}{\OB{e}}
\newcommand*{\ccS}{\OB{E}}
\newcommand*{\ccSem}[2][{}]{\OB{\ccS\sem[#1]{#2}}}
% value
\newcommand*{\val}{\OB{x}}
\newcommand*{\valy}{\OB{y}}
\newcommand*{\Val}{\OB{X}}
\newcommand*{\ValY}{\OB{Y}}
% lifted value
\newcommand*{\Bot}{\OB{\bot}}
%\newcommand*{\lval}{\OB{x}}
\newcommand*{\LVal}[1][\Val]{\OB{#1_{\Bot}}}
\newcommand*{\LValY}{\LVal[\ValY]}
% variational value
\newcommand*{\var}{\OB{v}}
\newcommand*{\Var}[1][\LVal]{\OB{V\,{#1}}}
\newcommand*{\VarY}{\Var[\LValY]}
% list
\newcommand*{\lst}{\OB{l}}
\newcommand*{\Lst}[1][\Val]{\OB{L\,{#1}}}
\newcommand*{\OLst}[1][\Val]{\OB{O\,{#1}}}
\newcommand*{\nil}{\OB{\varepsilon}}
% functions
\newcommand*{\sel}{\CCname{sel}}
\newcommand*{\osel}{\CCname{osel}}
\newcommand*{\lmap}{\CCname{map}}
\newcommand*{\omap}{\CCname{omap}}
\newcommand*{\hmap}{\CCname{hmap}}
\newcommand*{\lbind}{\CCname{bind}}
\newcommand*{\obind}{\CCname{obind}}
\newcommand*{\hbind}{\CCname{hbind}}
\newcommand*{\hzip}{\CCname{hzip}}
% generalization
\newcommand*{\TConsU}[1][(\cdot)]{\OB{U\,{#1}}}
\newcommand*{\TConsW}[1][(\cdot)]{\OB{W\,{#1}}}
\newcommand*{\TConsVU}{\OB{V\comp U}}
\newcommand*{\TConsUX}[1][\Val]{\TConsU[\Val]}
\newcommand*{\TConsWX}[1][\Val]{\TConsW[\Val]}
\newcommand*{\TConsUY}{\TConsU[\ValY]}
\newcommand*{\TConsWY}{\TConsW[\ValY]}
\newcommand*{\wsel}{\CCname{wsel}}

% other names

% abstraction
\newcommand*{\lam}[2]{\OB{\lambda#1.#2}}
% list operators
\newcommand*{\cons}{\mathbin{::}}
\newcommand*{\concat}{\mathbin{+\mkern-3mu+}}
% function composition
\newcommand*{\comp}{\circ}
% concrete list syntax
\newcommand*{\clst}[1]{\OB{\lbrack#1\rbrack}}


% beamer

% stepwise enumerate and itemize
\newenvironment{stepenum}{\begin{enumerate}[<+-| alert@+>]}{\end{enumerate}}
\newenvironment{stepitem}{\begin{itemize}[<+-| alert@+>]}{\end{itemize}}

% transparent overlays
%\setbeamercovered{transparent}

% themes
\usecolortheme{seahorse}
\usecolortheme{rose} % lily|orchid|rose
\useinnertheme{circles}

% remove navigation bar
%\beamertemplatenavigationsymbolsempty

% grid lines
% TODO: remove before presentation
\setbeamertemplate{background}[grid][step=2mm]

% slide numbers
\setbeamertemplate{footline}[frame number]

% blocks
%\setbeamertemplate{blocks}[rounded][shadow=true]
\addtobeamertemplate{block begin}{%
  \setlength{\abovedisplayskip}{0pt}%
  \setlength{\belowdisplayskip}{0pt}%
  \setlength{\abovedisplayshortskip}{0pt}%
  \setlength{\belowdisplayshortskip}{0pt}}{\vspace*{-1\baselineskip}}
\addtobeamertemplate{block example begin}{%
  \setlength{\abovedisplayskip}{0pt}%
  \setlength{\belowdisplayskip}{0pt}%
  \setlength{\abovedisplayshortskip}{0pt}%
  \setlength{\belowdisplayshortskip}{0pt}}{}
\addtobeamertemplate{block alerted begin}{%
  \setlength{\abovedisplayskip}{0pt}%
  \setlength{\belowdisplayskip}{0pt}%
  \setlength{\abovedisplayshortskip}{0pt}%
  \setlength{\belowdisplayshortskip}{0pt}}{}
\addtobeamertemplate{theorem begin}{%
  \setlength{\abovedisplayskip}{0pt}%
  \setlength{\belowdisplayskip}{0pt}%
  \setlength{\abovedisplayshortskip}{0pt}%
  \setlength{\belowdisplayshortskip}{0pt}}{\vspace*{\baselineskip}}
\addtobeamertemplate{proof begin}{}{\vspace*{\baselineskip}}
\addtobeamertemplate{itemize body begin}{%
  \setlength{\topsep}{0pt}%
  \setlength{\partopsep}{0pt}}{}

\title[]{A Formal Foundation for Variational Programming Using the Choice Calculus}
%\subtitle{}
\author{Spencer Hubbard}
\institute[OSU EECS]{
  School of Electrical Engineering and Computer Science\\
	Oregon State University
}
\date[6/13/16]{June 1, 2016}

\begin{document}
% front matter
\maketitle

\AtBeginSection[]
{
  \begin{frame}<beamer>[squeeze]
    \frametitle{Outline}
    \tableofcontents[currentsection]
  \end{frame}
}

% main matter
\section[Intro]{Introduction}

\begin{frame}<2>
  \frametitle{Introduction}
  %
  \begin{abstract}<only@1>
    ...
    We prove semantic equivalence rules for the formula choice calculus.
    ...
    We prove map and bind operations are sound for an implementation of variational lists.
    These proofs are written and verified in the language of the Coq proof assistant.
  \end{abstract}
  \note[item]<1>{welcome and thank you for attending}
  \note[item]<1>{abridged version of abstract}
  %
  \only<2>{\structure{Main Topics of Thesis}}
  \begin{enumerate}[<only@2->]
    \item Semantic equivalence rules for the formula choice calculus
    \item Sound map and bind operations for option-lists
    \item Formal verification with Coq proof assistant
  \end{enumerate}
  \note[item]<2>{\textbf{all claims} written and verified in Coq}
  \note[item]<2>{many details are omitted for lack of time}
\end{frame}

\begin{frame}
  \frametitle{Motivation}
  %
  \only<1>{\structure{Ren\'{e} Descartes, ``A Discourse on the Method''}}
  \begin{quote}<only@1>
    ...
    I judged that I was as prone to error as anyone else, and I rejected as false all the reasoning I had hitherto accepted as valid proof.
    ...
    while I was trying to think of all things being false in this way, it was necessarily the case that I, who was thinking them, had to be something;
    and observing this truth: I am thinking therefore I exist, was so secure and certain that it could not be shaken by any of the most extravagant suppositions of the skeptics, I judged that I could accept it without scruple, as the first principle of the philosophy I was seeking.
  \end{quote}
  \note[item]<1>{reference~\cite[p.~18]{descartes}}
  \note[item]<1>{motivational quote}
  \note[item]<1>{Descartes' famous statement of certainty ``cogito, ergo sum'' translated as ``I think, therefore I am''}
  \note[item]<1>{philosophy is method for obtaining knowledge from \textbf{secure foundations}}
  \note[item]<1>{purpose of thesis is to establish \textbf{secure foundations} for formula choice calculus and variational programming}
  %
  \only<2>{\structure{Use Cases}}
  \begin{itemize}[<only@2>]
    \item Projectional editing model for variational software
    \item TypeChef
    \item Software product lines (CIDE, CFJ, SPLlift, etc.)
    \item Many more
  \end{itemize}
  \note[item]<2>{semantic equivalence rules for formula choice calculus and variational data structures already being used without verification}
  \note[item]<2>{TypeChef is tool for parsing and type checking C code with preprocessor directives}
  \note[item]<2>{CIDE is IDE for software product lines}
  \note[item]<2>{CFJ is typed calculus for software product lines}
  \note[item]<2>{SPLlift is tool for inter-procedural data-flow analysis on software product lines}
  \note[item]<2>{other current research here at OSU and elsewhere}
\end{frame}

\section[FCC]{Formula Choice Calculus}

% customize metavariable for options
\renewcommand*{\dimMeta}{\dE}

\begin{frame}<2->
  \frametitle{Denotational Semantics of Choice Calculus}
  %
  \only<1>{\structure{Alfred North Whitehead, ``An Introduction to Mathematics''}}
  \begin{quote}<only@1>
    By relieving the brain of all unnecessary work, a good notation sets it free to concentrate on more advanced problems, and in effect increases the mental power of the race.
  \end{quote}
  \note[item]<1>{reference~\cite{whitehead}}
  %
  \begin{block}{Object Syntax}<only@2,3>
    \begin{syntax}
      \aE \in \aS &&& \text{Atom} \\
      \objE \in \objS
      & ::= & \objEmpty & \text{Empty} \\
      & | & \tr{\alert<3>{\objE}, \alert<3>{\objE}} & \text{Branch}
    \end{syntax}
  \end{block}
  \begin{block}{Expression Syntax}<only@2,3>
    \begin{syntax}
      \dE \in \dS &&& \text{Dimension} \\
      \ccE \in \ccS
      & ::= & \objEmpty & \text{Empty} \\
      & | & \tr{\alert<3>{\ccE}, \alert<3>{\ccE}} & \text{Branch} \\
      & | & \alert<3>{\chc{\ccE, \ccE}} & \text{Choice}
    \end{syntax}
  \end{block}
  \note[item]<3>{choice calculus is calculus for variation}
  \note[item]<3>{choice calculus is instantiated with object language}
  \note[item]<3>{object language is labeled binary trees}
  \note[item]<3>{differences are highlighted}
  \note[item]<3>{choices are \vocab{labeled} with \vocab{dimensions}}
  \note[item]<3>{subexpressions of choices are called \vocab{alternatives}}
  %
  \begin{description}[<only@4>][Configuration]
    \item[Tag] $\tagE \in \tagS = \set{\tagL, \tagR}$
    \item[Configuration] $\cE \in \cS = \dS \to \tagS$
  \end{description}
  \only<4>{\vspace*{-.5\baselineskip}}
  \begin{block}{Expression Semantics}<only@4>
    \begin{alignat*}{2}
      \mathrlap{\ccSem{\cdot} : \ccS \to \cS \to \objS} \\
      &\ccSem[\cE]{\objEmpty} &&= \objEmpty \\
      &\ccSem[\cE]{\tr{\ccE_1, \ccE_2}} &&=
      \tr{\ccSem[\cE]{\ccE_1}, \ccSem[\cE]{\ccE_2}} \\
      &\ccSem[\cE]{\chc{\ccE_1, \ccE_2}} &&=
      %\text{if } \cE\ \dE = \tagL
      %\text{ then } \ccSem[\cE]{\ccE_1}
      %\text{ else } \ccSem[\cE]{\ccE_2}
      \begin{cases}
        \ccSem[\cE]{\ccE_1}, & \text{if }\cE\ \dE = \tagL \\
        \ccSem[\cE]{\ccE_2}, & \text{otherwise}
      \end{cases}
    \end{alignat*}
  \end{block}
  \note[item]<4>{\vocab{tag} is either \vocab{left tag} or \vocab{right tag}}
  \note[item]<4>{\vocab{configuration} is function from dimensions to tags}
  \note[item]<4>{semantic domain is $\cS \to \objS$}
  %
  \begin{exampleblock}{Example}<only@5>
    Consider object language \objS\ for decimal notation.
    Then $123, 213 \in \objS$ and $\chc[\dE]{12,21}3 \in \ccS$ if $\dE \in \dS$.
    \\[10pt]
    $\ccSem[\cE]{\chc[\dE]{12,21}3} = 123$ if $\dE\fdef{\cE}\tagL$\\
    $\ccSem[\cE]{\chc[\dE]{12,21}3} = 213$ if $\dE\fdef{\cE}\tagR$
  \end{exampleblock}
  \begin{alertblock}{Expression Semantics Is Not Injective}<only@5>
    $\chc[\dE]{12,21}3 \ne \chc[\dE]{123,213}$ but $\ccSem[]{\chc[\dE]{12,21}3} = \ccSem[]{\chc[\dE]{123,213}}$
  \end{alertblock}
  \note[item]<5>{concrete example}
  \note[item]<5>{numeral sequence $123$ represents ``one hundred twenty three''}
  \note[item]<5>{numeral sequence $213$ represents ``two hundred thirteen''}
  \note[item]<5>{will return to issue of injectivity later}
\end{frame}

\begin{frame}
  \frametitle{Denotational Semantics of Formulas}
  %
  \begin{block}{Formula Syntax}<only@1>
    \begin{syntax}
      %\tagE \in \tagS &&& \text{Tag} \\
      %\dE \in \dS &&& \text{Dimension} \\
      \fE \in \fS
      & ::= & \tagE & \text{Tag} \\
      & | & \dE & \text{Dimension} \\
      & | & \Not \fE & \text{Complement} \\
      & | & \fE \lor \fE & \text{Join} \\
      & | & \fE \land \fE & \text{Meet}
    \end{syntax}
  \end{block}
  %
  \begin{onlyenv}<2>
    \structure{Tag Algebra}
    $$
      \begin{array}{c|c}
        \tagE & \Not \tagE \\
        \hline
        \tagL & \tagR \\
        \tagR & \tagL \\
      \end{array}
      \qquad
      \begin{array}{c|cc}
        \lor & \tagL & \tagR \\
        \hline
        \tagL & \tagL & \tagL \\
        \tagR & \tagL & \tagR \\
      \end{array}
      \qquad
      \begin{array}{c|cc}
        \land & \tagL & \tagR \\
        \hline
        \tagL & \tagL & \tagR \\
        \tagR & \tagR & \tagR \\
      \end{array}
    $$
  \end{onlyenv}
  \begin{block}{Formula Semantics}<only@2>
    \begin{alignat*}{2}
      &\mathrlap{\fSem{\cdot} : \fS \to \cS \to \tagS} \\
      &\fSem[\cE]{\tagE} &&= \tagE \\
      &\fSem[\cE]{\dE} &&= \cE\ \dE \\
      &\fSem[\cE]{\Not \fE} &&= \Not \fSem[\cE]{\fE} \\
      &\fSem[\cE]{\fE_1 \lor \fE_2} &&=
      \fSem[\cE]{\fE_1} \lor \fSem[\cE]{\fE_2} \\
      &\fSem[\cE]{\fE_1 \land \fE_2} &&=
      \fSem[\cE]{\fE_1} \land \fSem[\cE]{\fE_2}
    \end{alignat*}
  \end{block}
\end{frame}

\begin{frame}
  \frametitle{Formula Equivalence}
  %
  \begin{definition}[Formula Equivalence]
    Consider $\placeholder \equiv \placeholder \subseteq \fS \times \fS$ where $\fE \equiv \fE'$ iff $\fSem{\fE} = \fSem{\fE'}$.
    \begin{itemize}
      \item $\equiv$ is an equivalence relation
      \item $\equiv$ is called \vocab{formula equivalence}
      \item $\fE, \fE' \in \fS$ are called \vocab{semantically equivalent} if $\fE \equiv \fE'$
    \end{itemize}
  \end{definition}
  \begin{proof}[Informal Proof]
    $\fSem{\cdot}$ is well-defined with domain \fS\ and codomain $\cS \to \tagS$ and so the inverse images of elements in $\cS \to \tagS$ partition \fS.
  \end{proof}
\end{frame}

% customize metavariable for formula choices
\renewcommand*{\dimMeta}{\fE}

\begin{frame}
  \frametitle{Denotational Semantics of Formula Choice Calculus}
  %
  \begin{block}{Expression Syntax}<only@1,2>
    \begin{syntax}
      \fE \in \fS &&& \text{Formula} \\
      \ccE \in \ccS
      & ::= & \objEmpty & \text{Empty} \\
      & | & \tr{\ccE, \ccE} & \text{Branch} \\
      & | & \chc[\alert<2>{\fE}]{\ccE, \ccE} & \text{Choice}
    \end{syntax}
  \end{block}
  \begin{block}{Expression Semantics}<only@1,2>
    \begin{alignat*}{2}
      \mathrlap{\ccSem{\cdot} : \ccS \to \cS \to \objS} \\
      &\ccSem[\cE]{\objEmpty} &&= \objEmpty \\
      &\ccSem[\cE]{\tr{\ccE_1, \ccE_2}} &&=
      \tr{\ccSem[\cE]{\ccE_1}, \ccSem[\cE]{\ccE_2}} \\
      &\ccSem[\cE]{\chc{\ccE_1, \ccE_2}} &&=
      %\text{if } \cE\ \dE = \tagL
      %\text{ then } \ccSem[\cE]{\ccE_1}
      %\text{ else } \ccSem[\cE]{\ccE_2}
      \begin{cases}
        \ccSem[\cE]{\ccE_1}, & \text{if }\alert<2>{\fSem[\cE]{\fE}} = \tagL \\
        \ccSem[\cE]{\ccE_2}, & \text{otherwise}
      \end{cases}
    \end{alignat*}
  \end{block}
  \note[item]<2>{extension where choices labeled with formulas instead of dimensions}
  \note[item]<2>{changes are highlighted}
  \note[item]<2>{formulas are Boolean expressions in tags and dimensions}
  \note[item]<2>{\tagL\ is ``true'' and \tagR\ is ``false'' and \dE\ is Boolean variable}
  \note[item]<2>{can apply all rules of Boolean algebra to formulas}
  \note[item]<2>{semantic domain for formulas is $\cS \to \tagS$}
  \note[item]<2>{... which is truth assignment for dimensions}
  %
  \begin{exampleblock}{Example}<only@3>
    $\chc{12,21}3 \in \ccS$ if $\fE = \dE_1 \lor \dE_2$ and $\dE_1, \dE_2 \in \dS$
    \\[10pt]
    $\ccSem[\cE]{\chc{12,21}3} = 123$ if $\dE_1 \fdef{\cE} \tagL$ or $\dE_2\fdef{\cE}\tagL$\\
    $\ccSem[\cE]{\chc{12,21}3} = 213$ if $\dE_1 \fdef{\cE} \tagR$ and $\dE_2\fdef{\cE}\tagR$
  \end{exampleblock}
  \begin{alertblock}{Expression Semantics Is Not Injective}<only@3>
    $\chc{12,21}3 \ne \chc{123,213}$ but $\ccSem[]{\chc{12,21}3} = \ccSem[]{\chc{123,213}}$
  \end{alertblock}
  \note[item]<3>{concrete example}
  \note[item]<3>{same object language as before}
  \note[item]<3>{formulas provide more control over selection than dimensions}
  \note[item]<3>{formulas don't addressed issue of injectivity, in fact they add to it}
\end{frame}

\begin{frame}
  \frametitle{Expression Equivalence}
  %
  \begin{definition}[Expression Equivalence]
    Consider $\placeholder \equiv \placeholder \subseteq \ccS \times \ccS$ where $\ccE \equiv \ccE'$ iff $\ccSem{\ccE} = \ccSem{\ccE'}$.
    \begin{itemize}
      \item $\equiv$ is an equivalence relation
      \item $\equiv$ is called \vocab{expression equivalence}
      \item $\ccE, \ccE' \in \ccS$ are called \vocab{semantically equivalent} if $\ccE \equiv \ccE'$
    \end{itemize}
  \end{definition}
  \begin{proof}
    $\ccSem{\cdot}$ is well-defined with domain \ccS\ and codomain $\cS \to \objS$ and so the inverse images of elements in $\cS \to \objS$ partition \ccS.
  \end{proof}
  \note[item]{to address issue ... formal definition of expression equivalence}
  \note[item]{relation on expressions}
  \note[item]{$\equiv$ is equivalence relation ... here is proof}
  \note[item]{correspondence between partitions and equivalence relations}
  \note[item]{the equivalence relation corresponding to this partition is precisely $\equiv$}
  \note[item]{useful for replacing less efficient expressions with more efficient ones}
\end{frame}

\begin{frame}
  \frametitle{Syntactic Rules for Expression Equivalence}
  %
  \only<1>{\structure{Carl Friedrich Gauss, Doctoral Dissertation}}
  \begin{quote}<only@1>
    ...
    what is called a solution to an equation is, in reality, nothing but the reduction of the equation to prime equations---the solution is not exhibited but symbolized---and if you express a root of the equation $[\,x^2 = 2\,]$ by $[\,\sqrt{2}\,]$, you have not solved it nor done anything more than if you devised some symbol to denote a root of the equation $[\,x^2 - 2 = 0\,]$ and placed the root equal to this symbol
    ...
  \end{quote}
  \note[item]<1>{reference~\cite[p.~275]{ayoub80}}
  \note[item]<1>{Gauss' opinion of solutions of polynomial equations by radicals}
  \note[item]<1>{reduction similar to deriving $\ccE \equiv \ccE'$ without unfolding $\equiv$}
  \note[item]<1>{can derive $\ccE \equiv \ccE'$ without unfolding $\equiv$ using syntactic rules}
  \note[item]<1>{use cases apply these rules}
  \note[item]<1>{most rules are omitted for lack of time}
  %
  \begin{onlyenv}<2>
    \begin{block}{Expression Equivalence Rules}
      \vspace*{.5\baselineskip}
      \begin{mathpar}
        \inferrule[\rRefl]
          {}
          { \ccE \equiv \ccE }
          
        \inferrule[\rSymm]
          { \ccE \equiv \ccE' }
          { \ccE' \equiv \ccE }
  
        \inferrule[\rTran]
          { \ccE_1 \equiv \ccE_2 \\
            \ccE_2 \equiv \ccE_3 }
          { \ccE_1 \equiv \ccE_3 }
      \end{mathpar}
    \end{block}
    \begin{proof}
      $\equiv$ is an equivalence relation.
    \end{proof}
  \end{onlyenv}
  %
  \begin{onlyenv}<3>
    \begin{block}{Choice Congruence Rule}
      \vspace*{.5\baselineskip}
      \begin{mathpar}
        \inferrule[\rCCong]
          { \ccE_1 \equiv \ccE_1' \\
            \ccE_2 \equiv \ccE_2' \\
            \fE \equiv \fE' }
          { \chc{\ccE_1, \ccE_2}
            \equiv
            \chc[\fE']{\ccE_1', \ccE_2'} }
      \end{mathpar}
    \end{block}
    \begin{proof}[Proof (Idea)]
      Unfold definition of $\equiv$.
    \end{proof}
  \end{onlyenv}
  %
  \begin{onlyenv}<4>
    \begin{block}{Choice Transposition Rule}
      \vspace*{.5\baselineskip}
      \begin{mathpar}
        \inferrule[\rCTrans]
          {}
          { \chc{\ccE_1, \ccE_2}
            \equiv
            \chc[\Not \fE]{\ccE_2, \ccE_1} }
      \end{mathpar}
    \end{block}
    \begin{proof}[Proof (Idea)]
      Unfold definition of $\equiv$.
    \end{proof}
  \end{onlyenv}
  \note[item]<4>{semantics of choice is invariant under transposition of alternatives and complementation of label}
  \note[item]<4>{useful for deriving rules from other rules}
  \note[item]<4>{provide meaningful proofs ... show relationships}
  \note[item]<4>{avoid duplication of logic (and code) in proofs}
  %
  \begin{onlyenv}<5>
    \begin{block}{Choice Simplification Rules}
      \vspace*{.5\baselineskip}
      \begin{mathpar}
        \inferrule[\rCIdemp]
          {}
          { \chc{\ccE, \ccE} \equiv \ccE }
        
        \inferrule[\rCL]
          {}
          { \chc[\tagL]{\ccE_1, \ccE_2} \equiv \ccE_1 }
        
        \inferrule[\rCR]
          {}
          { \chc[\tagR]{\ccE_1, \ccE_2} \equiv \ccE_2 }
      \end{mathpar}
    \end{block}
    \begin{proof}
      Show \rCIdemp\ and \rCL\ by unfolding definition of $\equiv$ and derive \rCR\ from \rCL\ as follows:
      \begin{align*}
        \chc[\tagR]{\ccE_1, \ccE_2}
        &\equiv \chc[\Not \tagR]{\ccE_2, \ccE_1}
        &&\text{by \rCTrans}\\
        &\equiv \chc[\tagL]{\ccE_2, \ccE_1}
        &&\text{by }\Not \tagR \equiv \tagL\\
        &\equiv \ccE_2
        &&\text{by \rCL}
        \qedhere
      \end{align*}
    \end{proof}
  \end{onlyenv}
  \note[item]<5>{left-to-right: replace choice with alternative}
  %
  \begin{onlyenv}<6>
    \begin{block}{Formula Choice Rules}
      \vspace*{.5\baselineskip}
      \begin{mathpar}
        \inferrule[\rCJoin]
          {}
          { \chc[\fE_1]{
              \ccE_1,
              \chc[\fE_2]{\ccE_1, \ccE_2}}
            \equiv
            \chc[(\fE_1 \lor \fE_2)]{\ccE_1, \ccE_2} }

        \inferrule[\rCMeet]
          {}
          { \chc[\fE_1]{
              \chc[\fE_2]{\ccE_1, \ccE_2},
              \ccE_2}
            \equiv
            \chc[(\fE_1 \land \fE_2)]{\ccE_1, \ccE_2} }
      \end{mathpar}
    \end{block}
    \begin{proof}
      Show \rCJoin\ by unfolding definition of $\equiv$ and derive \rCMeet\ from \rCJoin\ as follows:
      \begin{align*}
        \chc[\fE_1]{
          \chc[\fE_2]{\ccE_1, \ccE_2},
          \ccE_2}
        &\equiv
        \chc[\Not\fE_1]{
          \ccE_2,
          \chc[\Not\fE_2]{\ccE_2, \ccE_1}}
        &&\text{by \rCTrans} \\
        &\equiv \chc[(\Not \fE_1 \lor \Not \fE_2)]{\ccE_2, \ccE_1}
        &&\text{by \rCJoin}\\
        &\equiv \chc[\Not(\fE_1 \land \fE_2)]{\ccE_2, \ccE_1}
        &&\text{by De Morgan's law}\\
        &\equiv \chc[(\fE_1 \land \fE_2)]{\ccE_1, \ccE_2}
        &&\text{by \rCTrans}
        \qedhere
      \end{align*}
    \end{proof}
  \end{onlyenv}
  \note[item]<6>{left-to-right: remove redundant alternatives in nested choices}
  %
  \begin{onlyenv}<7>
    \begin{block}{Choice Merge Rules}
      \vspace*{.5\baselineskip}
      \begin{mathpar}
        \inferrule[\rCCMergeL]
          {}
          { \chc{
              \chc{\ccE_1, \ccE_2},
              \ccE_3}
            \equiv
            \chc{\ccE_1, \ccE_3} }
        
        \inferrule[\rCCMergeR]
          {}
          { \chc{
              \ccE_1,
              \chc{\ccE_2, \ccE_3}}
            \equiv
            \chc{\ccE_1, \ccE_3} }
      \end{mathpar}
    \end{block}
    \begin{proof}
      Show \rCCMergeL\ by unfolding definition of $\equiv$ and derive \rCCMergeR\ from \rCCMergeL\ as follows:
      \begin{align*}
        \chc{\ccE_1, \chc{\ccE_2, \ccE_3}}
        &\equiv
        \chc[\Not\fE]{
          \chc[\Not\fE]{\ccE_3, \ccE_2},
          \ccE_1}
        &&\text{by \rCTrans} \\
        &\equiv \chc[\Not\fE]{\ccE_3, \ccE_1}
        &&\text{by \rCCMergeL} \\
        &\equiv \chc{\ccE_1, \ccE_3}
        &&\text{by \rCTrans}
        \qedhere
      \end{align*}
    \end{proof}
  \end{onlyenv}
  \note[item]<7>{left-to-right: remove unselectable alternatives in nested choices}
  \note[item]<7>{similar pattern to proofs for other rules}
  %
  \begin{onlyenv}<8>
    \begin{block}{Choice Commutation Rules}
      \vspace*{.5\baselineskip}
      \begin{mathpar}
        \inferrule[\rCCSwapL]
          {}
          { \chc[\fE_1]{
              \chc[\fE_2]{\ccE_1, \ccE_2},
              \ccE_3}
            \equiv
            \chc[\fE_2]{
              \chc[\fE_1]{\ccE_1, \ccE_3},
              \chc[\fE_1]{\ccE_2, \ccE_3}} }
        
        \inferrule[\rCCSwapR]
          {}
          { \chc[\fE_1]{
              \ccE_1,
              \chc[\fE_2]{\ccE_2, \ccE_3}}
            \equiv
            \chc[\fE_2]{
              \chc[\fE_1]{\ccE_1, \ccE_2},
              \chc[\fE_1]{\ccE_1, \ccE_3}} }
      \end{mathpar}
    \end{block}
    \begin{proof}
      Show \rCCSwapL\ by unfolding definition of $\equiv$ and derive \rCCSwapR\ from \rCCSwapL\ as follows:
      \begin{align*}
        \chc[\fE_1]{
          \ccE_1,
          \chc[\fE_2]{\ccE_2, \ccE_3}}
        &\equiv
        \chc[\Not\fE_1]{
          \chc[\fE_2]{\ccE_2, \ccE_3},
          \ccE_1}
        &&\text{by \rCTrans} \\
        &\equiv
        \chc[\fE_2]{
          \chc[\Not\fE_1]{\ccE_2, \ccE_1},
          \chc[\Not\fE_1]{\ccE_3, \ccE_1}}
        &&\text{by \rCCSwapL} \\
        &\equiv
        \chc[\fE_2]{
          \chc[\fE_1]{\ccE_1, \ccE_2},
          \chc[\fE_1]{\ccE_1, \ccE_3}}
        &&\text{by \rCTrans}
        \qedhere
      \end{align*}
    \end{proof}
  \end{onlyenv}
  \note[item]<8>{right-to-left: remove redundant alternatives in nested choices}
\end{frame}

\begin{frame}[fragile]
  \frametitle{Formal Verification}
  \setlength{\abovedisplayskip}{.4\baselineskip}%
  \setlength{\belowdisplayskip}{.4\baselineskip}%
  \structure{Informal Proof of \rCCMergeR}
  \begin{align*}
    \chc{\ccE_1, \chc{\ccE_2, \ccE_3}}
    &\equiv \chc[\Not\fE]{\chc[\Not\fE]{\ccE_3, \ccE_2}, \ccE_1}
    &&\text{by \rCTrans} \\
    &\equiv \chc[\Not\fE]{\ccE_3, \ccE_1}
    &&\text{by \rCCMergeL} \\
    &\equiv \chc{\ccE_1, \ccE_3}
    &&\text{by \rCTrans}
    \qedhere
  \end{align*}
  \structure{Formal Statement and Proof of \rCCMergeR\ in Coq}
  \begin{lstlisting}[gobble=4]
    Theorem CCMergeR : forall (f : Formula) (l r e : CC),
                       Chc f l (Chc f e r) =E= Chc f l r.
    Proof.
      intros f l r e.
      rewrite -> ChcCongR with (r' := Chc (~ f) r e) by apply ChcTrans.
      rewrite -> ChcTrans.
      rewrite -> CCMergeL.
      rewrite <- CCTrans.
      reflexivity.
    Qed.
  \end{lstlisting}
  \note[item]{informal and formal proofs are similar}
\end{frame}

\begin{frame}
  \frametitle{Generalization}
  %
  \begin{onlyenv}<1>
    \structure{How to Add Alternatives to Choices}
    \begin{enumerate}
      \item Add alternatives to syntax of choices
      \item Add tags to set of tags
      \item Add cases to semantics of choices
    \end{enumerate}
    \begin{alertblock}{Limitation}
      Set of tags must form a Boolean algebra but there are \textbf{no} Boolean algebras with odd numbers of elements.
      Therefore, we \textbf{cannot} add an odd number of alternatives to choices.
    \end{alertblock}
  \end{onlyenv}
  \note[item]<1>{natural way to extend choices}
  \note[item]<1>{proof of limitation would take us too far afield}
  %
  \begin{onlyenv}<2>
    \structure{Tag Group}
    $$
      \begin{array}{c|cc}
        \vartriangle & \tagL & \tagR \\
        \hline
        \tagL & \tagR & \tagL \\
        \tagR & \tagL & \tagR \\
      \end{array}
    $$
    \TODO{fill out}
  \end{onlyenv}
  \note[item]<2>{$\tagE_1 \vartriangle \tagE_2 = (\tagE_1 \land \Not \tagE_2) \lor (\tagE_2 \land \Not \tagE_1)$}
  \note[item]<2>{$\vartriangle$ is \vocab{symmetric difference} (or \vocab{exclusive or})}
  \note[item]<2>{\tagR\ is the identity element}
  %
  \begin{onlyenv}<3>
    \structure{Tag Algebra with Four Tags}
    $$
      \begin{array}{c|c}
        \tagE & \Not \tagE \\
        \hline
        \tagLL & \tagRR \\
        \tagLR & \tagRL \\
        \tagRL & \tagLR \\
        \tagRR & \tagLL \\
      \end{array}
      \qquad
      \begin{array}{c|cccc}
        \lor & \tagLL & \tagLR & \tagRL & \tagRR \\
        \hline
        \tagLL & \tagLL & \tagLL & \tagLL & \tagLL \\
        \tagLR & \tagLL & \tagLR & \tagLL & \tagLR \\
        \tagRL & \tagLL & \tagLL & \tagRL & \tagRL \\
        \tagRR & \tagLL & \tagLR & \tagRL & \tagRR \\
      \end{array}
      \qquad
      \begin{array}{c|cccc}
        \land & \tagLL & \tagLR & \tagRL & \tagRR \\
        \hline
        \tagLL & \tagLL & \tagLR & \tagRL & \tagRR \\
        \tagLR & \tagLR & \tagLR & \tagRR & \tagRR \\
        \tagRL & \tagRL & \tagRR & \tagRL & \tagRR \\
        \tagRR & \tagRR & \tagRR & \tagRR & \tagRR \\
      \end{array}
    $$
    \structure{Example Rule Change}
    \begin{mathpar}
      \inferrule[\rCCMerge]
        { \ccE_i = \chc{\ccE_{i1}, \ccE_{i2}, \ccE_{i3}, \ccE_{i4}} \\
          \Forall i \in \set{1, 2, 3, 4} }
        { \chc{\ccE_1, \ccE_2, \ccE_3, \ccE_4}
          \equiv
          \chc{\ccE_{11}, \ccE_{22}, \ccE_{33}, \ccE_{44}} }
    \end{mathpar}
  \end{onlyenv}
  \note[item]<3>{example with four alternatives}
  \note[item]<3>{Boolean algebra on four tags}
\end{frame}

\section[VP]{Variational Programming}

\begin{frame}
  \frametitle{Variational Value}
  %
  \begin{block}{Value Syntax}<only@1>
    \begin{syntax}
      \val \in \Val &&& \text{Value} \\
      \var \in \Var[\Val]
      & ::= & \val & \text{One} \\
      & | & \chc{\var, \var} & \text{Choice}
    \end{syntax}
  \end{block}
  \begin{block}{Selection Function}<only@1>
    \begin{alignat*}{2}
      &\mathrlap{\sel : \cS \to \Var[\Val] \to \Val} \\
      &\sel\ \cE\ \val &&= \val \\
      &\sel\ \cE\ \chc{\var_1, \var_2} &&=
      \begin{cases}
        \sel\ \cE\ \var_1, & \text{if }\fSem[\cE]{\fE} = \tagL \\
        \sel\ \cE\ \var_2, & \text{otherwise}
      \end{cases}
    \end{alignat*}
  \end{block}
  \note[item]<1>{arbitrary types are named \Val\ and \ValY}
  \note[item]<1>{naming convention established by Ren\'{e} Descartes in ``La G\'{e}om\'{e}trie''}
  \note[item]<1>{letters near end of alphabet for variables}
  \note[item]<1>{letters near beginning or middle of alphabet for constants}
  \note[item]<1>{variational values are choice calculus instantiated with trivial object language}
  \note[item]<1>{no construct for branches}
  \note[item]<1>{\sel\ implements expression semantics}
  %
  \begin{exampleblock}{Example}<only@2>
    $\chc[\dE_1]{1,\chc[\dE_2]{2,3}} \in \Var[\mathbb{Z}]$ if $\dE_1, \dE_2 \in \dS$\\
    \begin{align*}
      \sel\ \cE\ \chc[\dE_1]{1,\chc[\dE_2]{2,3}}
      &= \sel\ \cE\ \chc[\dE_2]{2,3} &&\text{if } \dE_1 \fdef{c} \tagR \\
      &= \sel\ \cE\ 2 &&\text{if } \dE_2 \fdef{c} \tagL \\
      &= 2 \\
      \sel\ \cE\ \chc[\dE_1]{1,\chc[\dE_2]{2,3}}
      &= \sel\ \cE\ \chc[\dE_2]{2,3} &&\text{if } \dE_1 \fdef{c} \tagR \\
      &= \sel\ \cE\ 3 &&\text{if } \dE_2 \fdef{c} \tagR \\
      &= 3
    \end{align*}
  \end{exampleblock}
  \note[item]<2>{$\mathbb{Z}$ is integers}
  \note[item]<2>{can select the leftmost alternative by $\dE_1 \fdef{c} \tagL$}
\end{frame}

\begin{frame}[fragile]
  \frametitle{Variational Function}
  %
  \begin{definition}[Soundness]
    For arbitrary functions $\varphi : \Var[\Val] \to \Var[\ValY]$ and $\psi : \Val \to \ValY$, say
    \begin{itemize}
      \item $\varphi$ and $\psi$ are \vocab{sound} \alert<2>{at $\cE \in \cS$} iff $\sel\ \cE \comp \varphi = \psi \comp \sel\ \cE$, i.e., TFDC:
      $$
        \begin{tikzcd}
          \Var[\Val] \ar[d, "\sel\ \cE"'] \ar[r, "\varphi"] \ar[dr, dashed] &
          \Var[\ValY] \ar[d, "\sel\ \cE"] \\
          \Val \ar[r, "\psi"'] & \ValY
        \end{tikzcd}
      $$
      \item $\varphi$ and $\psi$ are \vocab{sound} iff they are sound \alert<2>{at every configuration}
    \end{itemize}
  \end{definition}
  \note[item]<2>{$\varphi$ on variational values and $\psi$ on plain values}
  \note[item]<2>{$\varphi$ does same thing as $\psi$ to selected variants}
  \note[item]<2>{$\varphi$ does same thing as $\psi$ to all variants}
  \note[item]<2>{TFDC means ``the following diagram commutes''}
  \note[item]<2>{\cE\ is universally qualified in second definition}
\end{frame}

\begin{frame}
  \frametitle{Variational List}
  %
  \begin{onlyenv}<1>
  \begin{block}{List Syntax}
    \begin{syntax}
      \val \in \Val &&& \text{Element} \\
      \lst \in \Lst
      & ::= & \nil & \text{Nil} \\
      & | & \val \cons \lst & \text{Cons}
    \end{syntax}
  \end{block}
  \vspace*{-.5\baselineskip}
  \begin{description}[Option-List]
    \item[Option] $\LVal = \Val \cup \set{\Bot}$
    \item[Option-List] $\OLst = \Lst[(\Var)]$
  \end{description}
  \vspace*{-.5\baselineskip}
  \begin{block}{Selection Function}
    \begin{alignat*}{2}
      &\mathrlap{\osel : \cS \to \OLst \to \Lst} \\
      &\osel\ \cE\ \nil &&= \nil \\
      &\osel\ \cE\ (\var \cons \lst) &&=
      \begin{cases}
        \val \cons \osel\ \cE\ \lst, & \text{if }\sel\ \cE\ \var = \val \\
        \osel\ \cE\ \lst, & \text{otherwise}
      \end{cases}
    \end{alignat*}
  \end{block}
  \end{onlyenv}
  \note[item]<1>{standard list syntax}
  \note[item]<1>{options are either a value or a special bottom value}
  \note[item]<1>{option-lists are lists of variational options}
  \note[item]<1>{\osel\ discards bottom values}
  \note[item]<1>{option-lists implement variational lists}
  %
  \begin{exampleblock}{Example}<only@2>
    $1 \cons 3, 1 \cons 2 \cons 3 \in \Lst[\mathbb{Z}]$ and
    $1 \cons \chc[\dE]{\bot, 2} \cons 3 \in \OLst[\mathbb{Z}]$ if $\dE \in \dS$\\
    \begin{align*}
      \osel\ \cE\ (1 \cons \chc[\dE]{\bot, 2} \cons 3)
      &= 1 \cons \osel\ \cE\ (\chc[\dE]{\bot, 2} \cons 3) \\
      &= 1 \cons \osel\ \cE\ 3
      &&\text{if } \dE \fdef{\cE} \tagL \\
      &= 1 \cons 3 \\
      \osel\ \cE\ (1 \cons \chc[\dE]{\bot, 2} \cons 3)
      &= 1 \cons \osel\ \cE\ (\chc[\dE]{\bot, 2} \cons 3) \\
      &= 1 \cons 2 \cons \osel\ \cE\ 3
      &&\text{if } \dE \fdef{\cE} \tagR \\
      &= 1 \cons 2 \cons 3
    \end{align*}
  \end{exampleblock}
  \note[item]<2>{omitting last cons and nil in examples}
\end{frame}

\begin{frame}[fragile]
  \frametitle{Option-List Function}
  %
  \begin{definition}[Soundness]
    For arbitrary functions $\varphi : \alert<2>{\OLst[]}\Val \to \alert<2>{\OLst[]}\ValY$ and $\psi : \alert<2>{\Lst[]}\Val \to \alert<2>{\Lst[]}\ValY$, say
    \begin{itemize}
      \item $\varphi$ and $\psi$ are \vocab{sound} at $\cE \in \cS$ iff $\alert<2>{\osel}\ \cE \comp \varphi = \psi \comp \alert<2>{\osel}\ \cE$, i.e., TFDC:
      $$
        \begin{tikzcd}
          \alert<3>{\OLst[]}\Val \ar[d, "\alert<3>{\osel}\ \cE"'] \ar[r, "\varphi"] \ar[dr, dashed] &
          \alert<3>{\OLst[]}\ValY \ar[d, "\alert<3>{\osel}\ \cE"] \\
          \alert<3>{\Lst[]}\Val \ar[r, "\psi"'] & \alert<3>{\Lst[]}\ValY
        \end{tikzcd}
      $$
      \item $\varphi$ and $\psi$ are \vocab{sound} iff they are sound at every configuration
    \end{itemize}
  \end{definition}
\end{frame}

\begin{frame}
  \frametitle{Map}
  %
  \begin{block}{List Map Function}<only@1-3>
    \begin{alignat*}{2}
      &\mathrlap{\lmap : \alert<2>{(\Val \to \ValY)} \to \alert<3>{\Lst \to \Lst[\ValY]}} \\
      &\lmap\ \varphi\ \nil &&= \nil \\
      &\lmap\ \varphi\ (\val \cons \lst)
      &&= \varphi\ \val \cons \lmap\ \varphi\ \lst
    \end{alignat*}
  \end{block}
  \begin{block}{Option-List Map Function}<only@1-3>
    \begin{alignat*}{2}
      &\mathrlap{\omap : \alert<2>{(\Val \to \ValY)} \to \alert<3>{\OLst \to \OLst[\ValY]}} \\
      &\omap\ \varphi\ \nil &&= \nil \\
      &\omap\ \varphi\ (\var \cons \lst)
      &&= \hmap\ \varphi\ \var \cons \omap\ \varphi\ \lst \\[10pt]
      &\mathrlap{\hmap : (\Val \to \ValY) \to \Var \to \VarY}
    \end{alignat*}
  \end{block}
  \note[item]<3>{\lmap\ applies $\varphi$ to each list element}
  \note[item]<3>{\omap\ applies $\varphi$ to each $\ne \bot$ variant of option-list elements}
  \note[item]<3>{\hmap\ maps $\varphi$ over variational option}
  %
  \begin{exampleblock}{Example}<only@4>
    \vspace*{-1\baselineskip}
    \begin{align*}
      \lmap\ (\lam{x}{x^2})\ (1 \cons 2 \cons 3)
      &= 1 \cons \lmap\ (\lam{x}{x^2})\ (2 \cons 3) \\
      &= 1 \cons 4 \cons \lmap\ (\lam{x}{x^2})\ 3 \\
      &= 1 \cons 4 \cons 9
    \end{align*}
  \end{exampleblock}
  \begin{exampleblock}{Example}<only@4>
    \vspace*{-1\baselineskip}
    \begin{multline*}
      \omap\ (\lam{x}{x^2})\ (\chc[\dE]{1, 2} \cons \chc[\dE]{3, \bot}) \\
      \begin{aligned}
        &= \chc[\dE]{1, 4} \cons \omap\ (\lam{x}{x^2})\ \chc[\dE]{3, \bot} \\
        &= \chc[\dE]{1, 4} \cons \chc[\dE]{9, \bot}
      \end{aligned}
    \end{multline*}
  \end{exampleblock}
  \note[item]<4>{omitting last cons and nil in examples}
\end{frame}

\begin{frame}[fragile]
  \frametitle{Soundness of Map Function}
  %
  \begin{theorem}
    For each $\varphi : \Val \to \ValY$, \alert<2>{$\omap\ \varphi$} and \alert<2>{$\lmap\ \varphi$} are sound, i.e., for each $\cE \in \cS$, TFDC:
    $$
      \begin{tikzcd}
        \OLst \ar[d, "\osel\ \cE"'] \ar[r, "\omap\ \varphi"] \ar[dr, dashed] &
        \OLst[\ValY] \ar[d, "\osel\ \cE"] \\
        \Lst \ar[r, "\lmap\ \varphi"'] &
        \Lst[\ValY]
      \end{tikzcd}
    $$
  \end{theorem}
  \begin{proof}[Proof (Idea)]
    Show
    $
      (\osel\ \cE \comp \omap\ \varphi)\ \lst =
      (\lmap\ \varphi \comp \osel\ \cE)\ \lst
    $
    by structural induction on $\lst \in \OLst$.
  \end{proof}
  \note[item]<2>{proof is omitted for lack of time}
\end{frame}

\begin{frame}
  \frametitle{Bind}
  %
  \begin{block}{List Bind Function}<only@1,2>
    \begin{alignat*}{2}
      &\mathrlap{\lbind : (\Val \to \alert<2>{\Lst[]}\ValY) \to \Lst \to \Lst[\ValY]} \\
      &\lbind\ \varphi\ \nil &&= \nil \\
      &\lbind\ \varphi\ (\val\cons\lst)
      &&= \varphi\ \val \concat \lbind\ \varphi\ \lst
    \end{alignat*}
  \end{block}
  \begin{block}{Option-List Bind Function}<only@1,2>
    \begin{alignat*}{2}
      &\mathrlap{\obind : (\Val \to \alert<2>{\OLst[]}\ValY) \to \OLst \to \OLst[\ValY]} \\
      &\obind\ \varphi\ \nil &&= \nil \\
      &\obind\ \varphi\ (\var\cons\lst)
      &&= \hbind\ \varphi\ \var \concat \obind\ \varphi\ \lst \\[10pt]
      &\mathrlap{\hbind : (\Val \to \OLst[\ValY]) \to \Var \to \OLst[\ValY]}
    \end{alignat*}
  \end{block}
  \note[item]<2>{\lbind\ applies $\varphi$ to each list element and joins results}
  \note[item]<2>{\obind\ applies $\varphi$ to each $\ne \bot$ variant of option-list elements and joins results}
  \note[item]<2>{\hbind\ applies $\varphi$ to variational options, for choices it applies to alternatives and zips together results}
  \note[item]<2>{return types of first arguments differ}
  %
  \begin{exampleblock}{Example}<only@3>
    \vspace*{-1\baselineskip}
    \begin{multline*}
      \lbind\ (\lam{x}{x \cons x^2})\ (1 \cons 2 \cons 3) \\
      \begin{aligned}
        &= 1 \cons 1 \concat \lbind\ (\lam{x}{x \cons x^2})\ (2 \cons 3) \\
        &= 1 \cons 1 \cons 2 \cons 4 \concat \lbind\ (\lam{x}{x \cons x^2})\ 3 \\
        &= 1 \cons 1 \cons 2 \cons 4 \cons 3 \cons 9
      \end{aligned}
    \end{multline*}
  \end{exampleblock}
  \begin{exampleblock}{Example}<only@3>
    Let $\varphi = \lam{x}{\chc[\dE']{x, -x} \cons x^2}$. Then
    \begin{multline*}
      \obind\ \varphi\ (\chc[\dE]{1, 2} \cons \chc[\dE]{3, \bot}) \\
      \begin{aligned}
        &= \chc[\dE]{\chc[\dE']{1, -1}, \chc[\dE']{2, -2}} \cons \chc[\dE]{1, 4} \cons \obind\ \varphi\ \chc[\dE]{3, \bot} \\
        &= \chc[\dE]{\chc[\dE']{1, -1}, \chc[\dE']{2, -2}} \cons \chc[\dE]{1, 4} \cons \chc[\dE]{\chc[\dE']{3, -3}, \bot} \cons \chc[\dE]{9, \bot}
      \end{aligned}
    \end{multline*}
  \end{exampleblock}
  \note[item]<3>{omitting last cons and nil in examples}
\end{frame}

\begin{frame}[fragile]
  \frametitle{Soundness of Bind Function}
  %
  \begin{theorem}
    For each \alert<2>{$\varphi : \Val \to \OLst[\ValY]$} and \alert<2>{$\cE \in \cS$}, if \alert<2>{$\psi = \osel\ \cE \comp \varphi$}, then \alert<3>{$\obind\ \varphi$} and \alert<3>{$\lbind\ \psi$} are sound \alert<3>{at \cE}, i.e., TFDC:
    $$
      \begin{tikzcd}
        \OLst \ar[d, "\osel\ \cE"'] \ar[r, "\obind\ \varphi"] \ar[dr, dashed] &
        \OLst[\ValY] \ar[d, "\osel\ \cE"] \\
        \Lst \ar[r, "\lbind\ \psi"'] &
        \Lst[\ValY]
      \end{tikzcd}
    $$
  \end{theorem}
  \begin{proof}[Proof (Idea)]
    Show
    $
      (\osel\ \cE \comp \obind\ \varphi)\ \lst =
      (\lbind\ \psi \comp \osel\ \cE)\ \lst
    $
    by structural induction on $\lst \in \OLst$.
  \end{proof}
  \note[item]<3>{must compose $\varphi$ with $\osel\ \cE$ for type $\Val \to \Lst[\ValY]$}
  \note[item]<3>{no single function $\obind\ \varphi$ is sound with}
  \note[item]<3>{instead sound with family of functions indexed by \cE}
  \note[item]<3>{proof is omitted for lack of time}
\end{frame}

\begin{frame}<0>[fragile]
  \frametitle{Generalization}
  %
  \only<1>{\structure{Leonhard Euler, Elements of Algebra}}
  \begin{quote}<only@1>
    When it is required to extract the root of a negative number, a great difficulty arises; since there is no assignable number, the square of which would be a negative quantity
    ...
    we are lead to the idea of numbers, which from there nature are impossible; and therefore they are usually called imaginary quantities, because they exist merely in the imagination ...
    But notwithstanding this, these numbers present themselves to the mind; they exist in our imagination and we still have a sufficient idea of them; ...
    for this reason also, nothing prevents us from making use of these imaginary numbers, and employing them in calculations ...
  \end{quote}
  \note[item]<1>{reference~\cite[p.~42--43]{euler}}
  %
  \only<2>{\structure{Variational Data Structure}}
  \begin{enumerate}[<only@2>]
    \item type constructor $\TConsU[(\cdot)]$
    \item type constructor $\TConsW[(\cdot)]$ for \vocab{implementation} of $\Var[]\TConsU[(\cdot)]$
    \item function $\wsel : \cS \to \TConsW \to \TConsU$ for \vocab{selection}
  \end{enumerate}
  %
  \begin{onlyenv}<3-5>
    \setlength{\abovedisplayskip}{0pt}%
    \setlength{\belowdisplayskip}{0pt}%
    \structure{Soundness}\\
    For arbitrary functions $\varphi : \alert<4>{\TConsW[]}\Val \to \alert<4>{\TConsW[]}\ValY$ and $\psi : \alert<4>{\TConsU[]}\Val \to \alert<4>{\TConsU[]}\ValY$, say
    \begin{itemize}
      \item $\varphi$ and $\psi$ are \vocab{sound} at $\cE \in \cS$ iff $\alert<4>{\wsel}\ \cE \comp \varphi = \psi \comp \alert<4>{\wsel}\ \cE$, i.e., TFDC:
      $$
        \begin{tikzcd}
          \alert<5>{\TConsW[]}\Val \ar[d, "\alert<5>{\wsel}\ \cE"'] \ar[r, "\varphi"] \ar[dr, dashed] &
          \alert<5>{\TConsW[]}\ValY \ar[d, "\alert<5>{\wsel}\ \cE"] \\
          \alert<5>{\TConsU[]}\Val \ar[r, "\psi"'] & \alert<5>{\TConsW[]}\ValY
        \end{tikzcd}
      $$
      \item $\varphi$ and $\psi$ are \vocab{sound} iff they are sound at every $\cE \in \cS$
    \end{itemize}
  \end{onlyenv}
\end{frame}

% back matter
\section*{Questions}

\begin{frame}[c]
  \frametitle{Questions}
  %
  \Large\bfseries\centering Thank you!
  \note[item]<1>{\TODO{question about proof one rule from another} circular logic}
  \note[item]<1>{\TODO{question about even size Boolean algebra} elements must obey Boolean algebra laws}
\end{frame}

\section*{References}

\begin{frame}
  \frametitle{References}
  %
  \begin{thebibliography}{Whitehead}
    \bibitem[Ayoub]{ayoub80}
      Raymond Ayoub,
      \emph{Paolo Ruffini's Contributions to the Quintic}.
      Archive for History of Exact Sciences,
      vol.~23, no.~3, pp.~253--277, 1980.
    
    \bibitem[Descartes]{descartes}
      Ren{\'e} Descartes,
      \emph{{A Discourse on the Method}},
      translated by Ian Maclean.
      Oxford University Press, Oxford, 2006.

    \bibitem[Euler]{euler}
      Leonhard Euler,
      \emph{{Elements of Algebra}},
      translated by Rev.~John Hewlett.
      Springer-Verlag, New York, Berlin, Heidelberg, Tokyo, 1984.

    \bibitem[Whitehead]{whitehead}
      Alfred Whitehead,
      \emph{{An Introduction to Mathematics}}.
      Oxford University Press, Oxford, 1948.
  \end{thebibliography}
\end{frame}

\end{document}
