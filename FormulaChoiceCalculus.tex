\chapter{Formula Choice Calculus}
\label{ch:cc}

In this chapter, we establish semantic equivalence rules for the formula choice calculus.
The choice calculus is a metalanguage for describing variation in an arbitrary object language~\citep{EW11tosem}.
The formula choice calculus is an extension of the choice calculus where choices are labeled with formulas instead of dimensions~\citep{WO14gpce}.
The choice calculus is described in \secref{ccback}, formulas are described in \secref{f}, the formula choice calculus is described in \secref{fcc}, and a generalization is discussed in \secref{ccgen}.

\section{Choice Calculus}
\label{sec:ccback}

% customize metavariable for options
\renewcommand*{\dimMeta}{\dE}

Consider an object language \objS\ with abstract syntax described by the grammar in \figref{obj}.
Atoms represent symbols in the object language and have no internal structure.
The tree construct is binary for simplicity.
However, the syntax supports encoding of constructs with arbitrary arity, e.g., with right nested trees.

\begin{figure}[H]
  \onehalfspacing
  \begin{syntax}
    \aE \in \aS &&& \textit{Atom} \\
    \objE \in \objS
    & ::= & \objEmpty & \textit{Empty} \\
    & | & \tr{\objE, \objE} & \textit{Tree}
  \end{syntax}
  \caption{Object language syntax.}
  \label{fig:obj}
\end{figure}

The choice calculus is instantiated with the object language \objS.
The abstract syntax of choice calculus expressions is described by the grammar in \figref{ccsyn}.
The expressions in the choice construct are called \vocab{alternatives} and the dimension is called a \vocab{label}.

\begin{figure}[H]
  \onehalfspacing
  \begin{syntax}
    \dE \in \dS &&& \textit{Dimension} \\
    \ccE \in \ccS
    & ::= & \objEmpty & \textit{Empty} \\
    & | & \tr{\ccE, \ccE} & \textit{Tree} \\
    & | & \chc{\ccE, \ccE} & \textit{Choice}
  \end{syntax}
  \caption{Choice calculus syntax.}
  \label{fig:ccsyn}
\end{figure}

A \vocab{tag} is either the \vocab{left tag} \tagL\ or the \vocab{right tag} \tagR\ and a \vocab{configuration} is a (total) function from dimensions to tags.
Note that a function is total (or everywhere defined) by the mathematically accepted definition of a function and so we omit this hereafter.
The set of tags is $\tagS = \set{\tagL, \tagR}$ and the set of configurations is $\cS = \dS \to \tagS$.
The metavariables \tagE\ and \cE\ are used to represent arbitrary tags and configurations, respectively, unless qualified otherwise.

The semantic domain for expressions is the function domain $\cS \to \objS$, which is the set of functions from configurations to elements of the object language.
The semantic function for expressions is \ccSem{\cdot}, which is defined by the equations in \figref{ccsem}.
Note that we use juxtaposition to indicate function application, as in lambda calculus, e.g., we write $\cE\ \dE$ to indicate application of configuration \cE\ to dimension \dE.
In the last equation, note that if $\cE\ \dE \ne \tagL$, then $\cE\ \dE = \tagR$.
Elements of the object language in the image of \ccSem{\ccE} are called \vocab{variants} of \ccE\ and application of \ccSem{\ccE} is called \vocab{selection}.

\begin{figure}[H]
  \onehalfspacing
  \begin{alignat*}{2}
    \mathrlap{\ccSem{\cdot} : \ccS \to \cS \to \objS} \\
    &\ccSem[\cE]{\objEmpty} &&= \objEmpty \\
    &\ccSem[\cE]{\tr{\ccE_1, \ccE_2}} &&=
    \tr{\ccSem[\cE]{\ccE_1}, \ccSem[\cE]{\ccE_2}} \\
    &\ccSem[\cE]{\chc{\ccE_1, \ccE_2}} &&=
    \begin{cases}
      \ccSem[\cE]{\ccE_1}, & \text{if }\cE\ \dE = \tagL \\
      \ccSem[\cE]{\ccE_2}, & \text{otherwise}%(\cE\ \dE = \tagR)
    \end{cases}
  \end{alignat*}
  \caption{Choice calculus semantics.}
  \label{fig:ccsem}
\end{figure}

For example, consider the choice calculus instantiated with the object language of decimal notation.
Atoms of this object language are Arabic numerals, e.g., the Arabic numerals $1$, $2$, and $3$ are atoms.
Terms of this object language are right nested branches.
For convenience, we write the concrete syntax of terms as sequences of Arabic numerals, e.g., $123$ and $213$ are terms in the concrete syntax which correspond to the terms \tr[1]{\objEmpty, \tr[2]{\objEmpty, \tr[3]{\objEmpty, \objEmpty}}} and \tr[2]{\objEmpty, \tr[1]{\objEmpty, \tr[3]{\objEmpty, \objEmpty}}}, respectively, in the abstract syntax.
Suppose \dE\ is a dimension.
Then $\chc{12,21}3$ is a choice calculus expression.
This expression consists of a sequence with a choice followed by the term $3$.
The choice is labeled with the dimension \dE\ and the alternatives are the terms $12$ and $21$.
Moreover, for each configuration \cE, the semantics of this expression is the following:
%
\begin{inparaenum}[(1)]
  \item if $\cE\ \dE = \tagL$, then the variant $123$ is selected, and
  \item if $\cE\ \dE = \tagR$, then the variant $213$ is selected.
\end{inparaenum}

Choices can be extended to support more alternatives.
One way to do this is by adding tags to the set of tags and corresponding cases to the semantics of choices.
In this way, the number of alternatives is equal to the number of tags.
For example, to support three alternatives we add a third tag to the set of tags, extend the syntax of choices with a third alternative, and the semantics of choices with a third case.

\section{Formulas}
\label{sec:f}

In this section we present formulas in tags and dimensions.
We describe the syntax and semantics of formulas in \secref{fsem} and establish rules for deriving formula equivalence in \secref{feq}.

\subsection{Denotational Semantics}
\label{sec:fsem}

We begin with a ``semantics first''~\citep{EW11sle} description of formulas in tags.
The set \tagS\ forms an algebra with prefix unary operator ($\Not$), called \vocab{complement}, and infix binary operators ($\lor$) and ($\land$), called \vocab{join} and \vocab{meet}, respectively.
These operators are defined by the tables in \figref{tag}.
Note that \tagS\ is the Boolean algebra with two elements---up to isomorphism---where \tagL\ is ``true'' and \tagR\ is ``false.''

\begin{figure}[H]
  \onehalfspacing
  $$
    \begin{array}{c|c}
      \tagE & \Not \tagE \\
      \hline
      \tagL & \tagR \\
      \tagR & \tagL \\
    \end{array}
    \qquad
    \begin{array}{c|cc}
      \lor & \tagL & \tagR \\
      \hline
      \tagL & \tagL & \tagL \\
      \tagR & \tagL & \tagR \\
    \end{array}
    \qquad
    \begin{array}{c|cc}
      \land & \tagL & \tagR \\
      \hline
      \tagL & \tagL & \tagR \\
      \tagR & \tagR & \tagR \\
    \end{array}
  $$
  \caption{Boolean algebra on tags.}
  \label{fig:tag}
\end{figure}

The abstract syntax of formulas is described by the grammar in \figref{fsyn}.
Note that we reuse the symbols for the operators defined in the previous paragraph.
The semantic domain for formulas is the function domain $\cS \to \tagS$, which is the set of functions from configurations to tags.
The semantic function for formulas is \fSem{\cdot}, which is defined by the equations in \figref{fsem}.

\begin{figure}[H]
  \onehalfspacing
  \begin{syntax}
    \fE \in \fS
    & ::= & \tagE & \textit{Tag} \\
    & | & \dE & \textit{Dimension} \\
    & | & \Not \fE & \textit{Complement} \\
    & | & \fE \lor \fE & \textit{Join} \\
    & | & \fE \land \fE & \textit{Meet}
  \end{syntax}
  \caption{Formula syntax.}
  \label{fig:fsyn}
\end{figure}

\begin{figure}[H]
  \onehalfspacing
  \begin{alignat*}{2}
    &\mathrlap{\fSem{\cdot} : \fS \to \cS \to \tagS} \\
    &\fSem[\cE]{\tagE} &&= \tagE \\
    &\fSem[\cE]{\dE} &&= \cE\ \dE \\
    &\fSem[\cE]{\Not \fE} &&= \Not \fSem[\cE]{\fE} \\
    &\fSem[\cE]{\fE_1 \lor \fE_2} &&=
    \fSem[\cE]{\fE_1} \lor \fSem[\cE]{\fE_2} \\
    &\fSem[\cE]{\fE_1 \land \fE_2} &&=
    \fSem[\cE]{\fE_1} \land \fSem[\cE]{\fE_2}
  \end{alignat*}
  \caption{Formula semantics.}
  \label{fig:fsem}
\end{figure}

\subsection{Semantic Equivalence}
\label{sec:feq}

Before we define semantic equivalence, it is useful to recall the following elementary facts:
%
\begin{inparaenum}[(1)]
  \item Two functions are equal, by definition, if they have the same domain and codomain and their images agree for all elements in the domain.
  \item For any function, the inverse images of elements in the codomain partition the domain.
  \item Any partition defines a ``canonical'' equivalence relation where the parts of the partition correspond to the equivalence classes of the relation.
\end{inparaenum}

Let ($\equiv$) be a binary relation on formulas defined by $\fE \equiv \fE'$ if and only if $\fSem{\fE} = \fSem{\fE'}$.
Note that the relation ($\equiv$) is defined in terms of function equality.
Since \fSem{\cdot} is well-defined as a function from formulas to elements of the semantic domain, it follows from earlier remarks that the relation ($\equiv$) is an equivalence relation.
We refer to the relation ($\equiv$) as \vocab{semantic equivalence} and we call formulas \fE\ and $\fE'$ (\vocab{semantically}) \vocab{equivalent} if $\fE \equiv \fE'$.

In the remainder of this section, we establish some syntactic rules for deriving formula equivalence.
In practice, formula equivalence can be checked by solving an instance of the satisfiability problem, e.g., using a SAT solver, and this is justified by the syntactic rules.
%The rules are organized into three sets:
%\begin{inparaenum}[(1)]
%  \item formula equivalence rules,
%  \item formula congruence rules, and
%  \item algebraic rules.
%\end{inparaenum}
We prove these rules are correct in \appref{f-coq}.
%However, we do not prove these rules are complete, i.e., if $\fE \equiv \fE'$, then there is a derivation by the syntactic rules.
%The rules are expected to be complete because they cover every possible pairwise interaction between constructs in the syntax.
%Proving the completeness of a minimal set of rules is left to future work.

The formula equivalence rules are stated in \figref{fequiv}.
These rules state that formulas equivalence is reflexive, symmetric, and transitive.
These rules follow directly from the definition of an equivalence relation.

\begin{figure}[H]
  \onehalfspacing
  \begin{mathpar}
    \inferrule[\rReflE]
      {}
      { \fE \equiv \fE }
    
    \inferrule[\rSymmE]
      { \fE \equiv \fE' }
      { \fE' \equiv \fE }
    
    \inferrule[\rTranE]
      { \fE_1 \equiv \fE_2 \\
        \fE_2 \equiv \fE_3 }
      { \fE_1 \equiv \fE_3 }
  \end{mathpar}
  \caption{Formula equivalence rules.}
  \label{fig:fequiv}
\end{figure}

The formula congruence rules are stated in \figref{fcong}.
Note that the formula congruence rules are not independent of each other, e.g., \rJoinCong\ can be derived from \rJoinCongL\ and \rJoinCongR.

\begin{figure}[H]
  \onehalfspacing
  \begin{mathpar}
    \inferrule[\rCompCong]
      { \fE \equiv \fE' }
      { \Not\fE \equiv \Not\fE' }

    \inferrule[\rJoinCong]
      { \fE_1 \equiv \fE_1' \\
        \fE_2 \equiv \fE_2' }
      { \fE_1 \lor \fE_2
        \equiv
        \fE_1' \lor \fE_2' }

    \inferrule[\rJoinCongL]
      { \fE_1 \equiv \fE_1' }
      { \fE_1 \lor \fE_2
        \equiv
        \fE_1' \lor \fE_2 }

    \inferrule[\rJoinCongR]
      { \fE_2 \equiv \fE_2' }
      { \fE_1 \lor \fE_2
        \equiv
        \fE_1 \lor \fE_2' } \\

    \inferrule[\rMeetCong]
      { \fE_1 \equiv \fE_1' \\
        \fE_2 \equiv \fE_2' }
      { \fE_1 \land \fE_2
        \equiv
        \fE_1' \land \fE_2' }

    \inferrule[\rMeetCongL]
      { \fE_1 \equiv \fE_1' }
      { \fE_1 \land \fE_2
        \equiv
        \fE_1' \land \fE_2 }

    \inferrule[\rMeetCongR]
      { \fE_2 \equiv \fE_2' }
      { \fE_1 \land \fE_2
        \equiv
        \fE_1 \land \fE_2' }
  \end{mathpar}
  \caption{Formula congruence rules.}
  \label{fig:fcong}
\end{figure}

The algebraic rules are stated in \figref{alg}.
These are the usual rules of Boolean algebra expressed for formula equivalence.

\begin{figure}[H]
  \onehalfspacing
  \begin{mathpar}
    \inferrule[\rJoinComp]
      {}
      { \fE \lor \Not \fE \equiv \tagL }

    \inferrule[\rMeetComp]
      {}
      {\fE \land \Not \fE \equiv \tagR }

    \inferrule[\rJoinId]
      {}
      { \tagR \lor \fE \equiv \fE }

    \inferrule[\rMeetId]
      {}
      { \tagL \land \fE \equiv \fE } \\

    \inferrule[\rJoinIdemp]
      {}
      { \fE \lor \fE \equiv \fE }

    \inferrule[\rMeetIdemp]
      {}
      { \fE \land \fE \equiv \fE }

    \inferrule[\rJoinComm]
      {}
      { \fE_1 \lor \fE_2
        \equiv
        \fE_2 \lor \fE_1 }

    \inferrule[\rMeetComm]
      {}
      { \fE_1 \land \fE_2
        \equiv
        \fE_2 \land \fE_1 }

    \inferrule[\rJoinAssoc]
      {}
      { \fE_1 \lor (\fE_2 \lor \fE_3)
        \equiv
        (\fE_1 \lor \fE_2) \lor \fE_3 }

    \inferrule[\rMeetAssoc]
      {}
      { \fE_1 \land (\fE_2 \land \fE_3)
        \equiv
        (\fE_1 \land \fE_2) \land \fE_3 }

    \inferrule[\rJoinDist]
      {}
      { \fE_1 \lor (\fE_2 \land \fE_3)
        \equiv
        (\fE_1 \lor \fE_2) \land (\fE_1 \lor \fE_3) }

    \inferrule[\rMeetDist]
      {}
      { \fE_1 \land (\fE_2 \lor \fE_3)
        \equiv
        (\fE_1 \land \fE_2) \lor (\fE_1 \land \fE_3) }

    \inferrule[\rCompJoin]
      {}
      { \Not(\fE_1 \lor \fE_2)
        \equiv
        \Not \fE_1 \land \Not \fE_2 }

    \inferrule[\rCompMeet]
      {}
      { \Not(\fE_1 \land \fE_2)
        \equiv
        \Not \fE_1 \lor \Not \fE_2 }
  \end{mathpar}
  \caption{Algebraic rules.}
  \label{fig:alg}
\end{figure}

Note that the formula congruence and algebraic rules imply that the set of equivalence classes of formulas is isomorphic to the free Boolean algebra on the set of dimensions \dS.
The operations on equivalence classes are defined in terms of the operations on representative formulas and the formula congruence rules ensure that these operations are well-defined.

As an aside, consider the binary infix operator ($\vartriangle$) on the set of equivalence classes, called \vocab{symmetric difference} (or \vocab{exclusive or}).
For any representative formulas $\fE_1, \fE_2 \in \fS$, the symmetric difference $\fE_1 \vartriangle \fE_2$ is defined to be the equivalence class with representative formula $(\fE_1 \land \Not \fE_2) \lor (\fE_2 \land \Not \fE_1)$.
Note that the equivalence classes form a group with law of composition given by symmetric difference.
We revisit this idea in \secref{ccgen}.

\section{Formula Choice Calculus}
\label{sec:fcc}

% customize metavariable for formula choices
\renewcommand*{\dimMeta}{\fE}

In this section we present the formula choice calculus.
We describe the syntax and semantics of expressions in \secref{fccsem} and establish rules for deriving expression equivalence in \secref{fcceq}.

\subsection{Denotational Semantics}
\label{sec:fccsem}

The abstract syntax of expressions is described by the grammar in \figref{fccsyn}.
The semantic domain for expressions is the function domain $\cS \to \objS$, which is the set of functions from configurations to elements of the object language.
The semantic function for expressions is \ccSem{\cdot}, which is defined by the equations in \figref{fccsem}.
In the last equation, note that if $\fSem[\cE]{\fE} \ne \tagL$, then $\fSem[\cE]{\fE} = \tagR$.

\begin{figure}[H]
  \onehalfspacing
  \begin{syntax}
    \ccE \in \ccS
    & ::= & \objEmpty & \textit{Empty} \\
    & | & \tr{\ccE, \ccE} & \textit{Tree} \\
    & | & \chc{\ccE, \ccE} & \textit{Choice}
  \end{syntax}
  \caption{Formula choice calculus syntax.}
  \label{fig:fccsyn}
\end{figure}

\begin{figure}[H]
  \onehalfspacing
  \begin{alignat*}{2}
    &\mathrlap{\ccSem{\cdot} : \ccS \to \cS \to \objS} \\
    &\ccSem[\cE]{\objEmpty} &&= \objEmpty \\
    &\ccSem[\cE]{\tr{\ccE_1, \ccE_2}} &&=
    \tr{\ccSem[\cE]{\ccE_1}, \ccSem[\cE]{\ccE_2}} \\
    &\ccSem[\cE]{\chc{\ccE_1, \ccE_2}} &&=
    \begin{cases}
      \ccSem[\cE]{\ccE_1}, & \text{if }\fSem[\cE]{\fE} = \tagL \\
      \ccSem[\cE]{\ccE_2}, & \text{otherwise}%(\fSem[\cE]{\fE} = \tagR)
    \end{cases}
  \end{alignat*}
  \caption{Formula choice calculus semantics.}
  \label{fig:fccsem}
\end{figure}

For example, consider the formula choice calculus instantiated with the same object language as before, i.e., the object language of decimal notation.
Suppose $\dE_1$ and $\dE_2$ are dimensions.
Then $\chc[(\dE_1 \lor \dE_2)]{12,21}3$ is a formula choice calculus expression.
This expression consists of a similar sequence as before, except the choice is labeled with the formula $\dE_1 \lor \dE_2$.
Moreover, for each configuration \cE, the semantics of this expression is the following:
%
\begin{inparaenum}[(1)]
  \item if $\cE\ \dE_1 = \tagL$ or $\cE\ \dE_2 = \tagL$, then the variant $123$ is selected, and
  \item if $\cE\ \dE_1 = \tagR$ and $\cE\ \dE_2 = \tagR$, then the variant $213$ is selected.
\end{inparaenum}

\subsection{Semantic Equivalence}
\label{sec:fcceq}

Semantic equivalence for expressions is defined in the same way as it is for formulas.
Let ($\equiv$) be a binary relation on expressions defined by $\ccE \equiv \ccE'$ if and only if $\ccSem{\ccE} = \ccSem{\ccE'}$.
By the same reasoning as before, the relation ($\equiv$) is an equivalence relation.
We refer to the relation ($\equiv$) as \vocab{semantic equivalence} and we call expressions \ccE\ and $\ccE'$ (\vocab{semantically}) \vocab{equivalent} if $\ccE \equiv \ccE'$.

In the remainder of this section, we establish some syntactic rules for deriving expression equivalence.
We prove these rules are correct in \appref{fcc-coq}.
However, we do not prove these rules are complete, i.e., if $\ccE \equiv \ccE'$, then there is a derivation by these rules.
Proving the completeness of a minimal set of rules is left to future work.

All of the rules stated in this section can be proved directly from the definition of expression equivalence.
However, this often leads to duplication of logic among the proofs.
Moreover, these proofs are not very insightful.
To avoid duplication and to describe relationships among rules, we derive some rules from others---rather than directly from definitions.

The expression equivalence rules are stated in \figref{eequiv}.
These rules state that expression equivalence is reflexive, symmetric, and transitive.
These rules follow directly from the definition of an equivalence relation.

\begin{figure}[H]
  \onehalfspacing
  \begin{mathpar}
  \inferrule[\rReflE]
    {}
    { \ccE \equiv \ccE }
  
  \inferrule[\rSymmE]
    { \ccE \equiv \ccE' }
    { \ccE' \equiv \ccE }
  
  \inferrule[\rTranE]
    { \ccE_1 \equiv \ccE_2 \\
      \ccE_2 \equiv \ccE_3 }
    { \ccE_1 \equiv \ccE_3 }
  \end{mathpar}
  \caption{Expression equivalence rules.}
  \label{fig:eequiv}
\end{figure}

The choice transposition rule is stated in \figref{ctrans}.
This rule states that the semantics of a choice is invariant under transposition of its alternatives and complementation of its label.
%This rule describes the relationship between formulas and choices.
This rule allows us to derive some rules from others and it is used extensively in these derivations.
We often omit details of applying this rule for brevity.
We prove this rule directly from the definition of expression equivalence, see \appref{fcc-coq}.

\begin{figure}[H]
  \onehalfspacing
  \begin{mathpar}
    \inferrule[\rCTrans]
      {}
      { \chc{\ccE_1, \ccE_2}
        \equiv
        \chc[\Not \fE]{\ccE_2, \ccE_1} }
  \end{mathpar}
  \caption{Choice transposition rule.}
  \label{fig:ctrans}
\end{figure}

The object congruence rules are stated in \figref{ocong}.
Note that these rules are not independent of each other.
For example, \rOCong\ can be derived from the other two rules.
These rules do not refer to formula choices directly and so they do not differ from the corresponding rules for dimension choices in a meaningful way.
For completeness, we include proofs of these rules in \appref{fcc-coq}.

\begin{figure}[H]
  \onehalfspacing
  \begin{mathpar}
    \inferrule[\rOCong]
      { \ccE_1 \equiv \ccE_1' \\
        \ccE_2 \equiv \ccE_2' }
      { \tr{\ccE_1, \ccE_2}
        \equiv
        \tr{\ccE_1', \ccE_2'} }

    \inferrule[\rOCongL]
      { \ccE_1 \equiv \ccE_1' }
      { \tr{\ccE_1, \ccE_2}
        \equiv
        \tr{\ccE_1', \ccE_2} }

    \inferrule[\rOCongR]
      { \ccE_2 \equiv \ccE_2' }
      { \tr{\ccE_1, \ccE_2}
        \equiv
        \tr{\ccE_1, \ccE_2'} }
  \end{mathpar}
  \caption{Object congruence rules.}
  \label{fig:ocong}
\end{figure}

The choice congruence rules are stated in \figref{ccong}.
Note that these rules are not independent of each other.
For example, we show \rCCong\ directly from the other three rules in the proof of \thmref{ccong}.
Of course, we could also derive the other three rules directly from \rCCong.
The choice congruence rule for labels \rCCongF\ is used in derivations of some other rules.
We often omit details of applying this rule for brevity.

\begin{figure}[H]
  \onehalfspacing
  \begin{mathpar}
    \inferrule[\rCCong]
      { \ccE_1 \equiv \ccE_1' \\
        \ccE_2 \equiv \ccE_2' \\
        \fE \equiv \fE' }
      { \chc{\ccE_1, \ccE_2}
        \equiv
        \chc[\fE']{\ccE_1', \ccE_2'} } \\

    \inferrule[\rCCongF]
      { \fE \equiv \fE' }
      { \chc{\ccE_1, \ccE_2}
        \equiv
        \chc[\fE']{\ccE_1, \ccE_2} }

    \inferrule[\rCCongL]
      { \ccE_1 \equiv \ccE_1' }
      { \chc{\ccE_1, \ccE_2}
        \equiv
        \chc{\ccE_1', \ccE_2} }

    \inferrule[\rCCongR]
      { \ccE_2 \equiv \ccE_2' }
      { \chc{\ccE_1, \ccE_2}
        \equiv
        \chc{\ccE_1, \ccE_2'} }
  \end{mathpar}
  \caption{Choice congruence rules.}
  \label{fig:ccong}
\end{figure}

\begin{theorem}
  \label{thm:ccong}
  The choice congruence rules hold for all formulas $\fE, \fE' \in \fS$ and expressions $\ccE_1, \ccE_1', \ccE_2, \ccE_2' \in \ccS$.
\end{theorem}

\begin{proof}
  First, we show \rCCongF\ and \rCCongL\ directly from the definition of expression equivalence, see \appref{fcc-coq}.
  Next, we show \rCCongR\ by deriving the consequent from the antecedent and \rCCongL.
  The derivation is as follows.
  \begin{align*}
    \chc{\ccE_1, \ccE_2}
    &\equiv \chc[\Not \fE]{\ccE_2, \ccE_1} \\
    &\equiv \chc[\Not \fE]{\ccE_2', \ccE_1} \\
    &\equiv \chc{\ccE_1, \ccE_2'}
  \end{align*}
  Finally, \rCCong\ follows directly from the other three rules.
\end{proof}

Observe that converses of the congruence rules for choices do not hold, e.g., in general $\chc{\ccE_1, \ccE_2} \equiv \chc[\fE']{\ccE_1, \ccE_2}$ is not sufficient for $\fE \equiv \fE'$ and $\chc{\ccE_1, \ccE_2} \equiv \chc{\ccE_1', \ccE_2}$ is not sufficient for $\ccE_1 \equiv \ccE_1'$.
For a counterexample to the converse of \rCCongF, consider $\chc[\tagL]{\aE, \aE} \equiv \aE \equiv \chc[\tagR]{\aE, \aE}$ but $\tagL \not\equiv \tagR$.
For a counterexample to the converse of \rCCongL, consider $\chc[\tagR]{\aE, \aE} \equiv \aE \equiv \chc[\tagR]{\aE', \aE}$ but $\aE \not\equiv \aE'$ if $\aE \ne \aE'$.
There is a similar counterexample to the converse of \rCCongR.

The choice simplification rules are stated in \figref{csimpl}.
These rules describe certain cases where the semantics of a choice is equivalent to one of its alternatives.

\begin{figure}[H]
  \begin{mathpar}
    \inferrule[\rCIdemp]
      {}
      { \chc{\ccE, \ccE} \equiv \ccE }

    \inferrule[\rCL]
      {}
      { \chc[\tagL]{\ccE_1, \ccE_2} \equiv \ccE_1 }

    \inferrule[\rCR]
      {}
      { \chc[\tagR]{\ccE_1, \ccE_2} \equiv \ccE_2 }
  \end{mathpar}
  \caption{Choice simplification rules.}
  \label{fig:csimpl}
\end{figure}

\begin{theorem}
  \label{thm:csimpl}
  The choice simplification rules hold for all formulas $\fE \in \fS$ and expressions $\ccE, \ccE_1, \ccE_2 \in \ccS$.
\end{theorem}

\begin{proof}
  First, we show \rCIdemp\ and \rCL\ directly from the definition of expression equivalence, see \appref{fcc-coq}.
  Next, we show \rCR\ by \rCL\ and $\Not \tagR \equiv \tagL$, which can be shown from the formula congruence and algebraic rules, see \appref{f-coq}.
  The derivation of \rCR\ is as follows.
  \begin{align*}
    \chc[\tagR]{\ccE_1, \ccE_2}
    &\equiv \chc[\Not \tagR]{\ccE_2, \ccE_1} \\
    &\equiv \chc[\tagL]{\ccE_2, \ccE_1} \\
    &\equiv \ccE_2 \qedhere
  \end{align*}
\end{proof}

The formula choice rules are stated in \figref{fc}.
Redundant alternatives in nested choices can be eliminated while preserving semantics by applying these rules from left-to-right.

\begin{figure}[H]
  \begin{mathpar}
    \inferrule[\rCJoin]
      {}
      { \chc[\fE_1]{
          \ccE_1,
          \chc[\fE_2]{\ccE_1, \ccE_2}}
        \equiv
        \chc[(\fE_1 \lor \fE_2)]{\ccE_1, \ccE_2} }

    \inferrule[\rCMeet]
      {}
      { \chc[\fE_1]{
          \chc[\fE_2]{\ccE_1, \ccE_2},
          \ccE_2}
        \equiv
        \chc[(\fE_1 \land \fE_2)]{\ccE_1, \ccE_2} }

    \inferrule[\rCJoinC]
      {}
      { \chc[\fE_1]{
          \ccE_1,
          \chc[\fE_2]{\ccE_2, \ccE_1}}
        \equiv
        \chc[(\fE_1 \lor \Not \fE_2)]{\ccE_1, \ccE_2} }

    \inferrule[\rCMeetC]
      {}
      { \chc[\fE_1]{
          \chc[\fE_2]{\ccE_2, \ccE_1},
          \ccE_2}
        \equiv
        \chc[(\fE_1 \land \Not \fE_2)]{\ccE_1, \ccE_2} }
  \end{mathpar}
  \caption{Formula choice rules.}
  \label{fig:fc}
\end{figure}

\begin{theorem}
  \label{thm:fc}
  The formula choice rules hold for all formulas $\fE_1, \fE_2 \in \fS$ and expressions $\ccE_1, \ccE_2 \in \ccS$.
\end{theorem}

\begin{proof}
  First, we show \rCJoin\ directly from the definition of expression equivalence, see \appref{fcc-coq}.
  Next, we show \rCMeet\ by \rCJoin\ and \rCompMeet.
  The derivation of \rCMeet\ is as follows.
  \begin{align*}
    \chc[\fE_1]{
      \chc[\fE_2]{\ccE_1, \ccE_2},
      \ccE_2} &\equiv
    \chc[\Not\fE_1]{
      \ccE_2,
      \chc[\Not\fE_2]{\ccE_2, \ccE_1}} \\
    &\equiv \chc[(\Not \fE_1 \lor \Not \fE_2)]{\ccE_2, \ccE_1} \\
    &\equiv \chc[\Not(\fE_1 \land \fE_2)]{\ccE_2, \ccE_1} \\
    &\equiv \chc[(\fE_1 \land \fE_2)]{\ccE_1, \ccE_2}
  \end{align*}
  Finally, we show \rCJoinC\ and \rCMeetC\ by \rCJoin\ and \rCMeet, respectively.
  The derivation of \rCJoinC\ is as follows.
  \begin{align*}
    \chc[\fE_1]{
      \ccE_1,
      \chc[\fE_2]{\ccE_2, \ccE_1}} &\equiv
    \chc[\fE_1]{
      \ccE_1,
      \chc[\Not \fE_2]{\ccE_1, \ccE_2}} \\
    &\equiv \chc[(\fE_1 \lor \Not \fE_2)]{\ccE_1, \ccE_2}
  \end{align*}
  The derivation of \rCMeetC\ is as follows.
  \begin{align*}
    \chc[\fE_1]{
      \chc[\fE_2]{\ccE_2, \ccE_1},
      \ccE_2} &\equiv
    \chc[\fE_1]{
      \chc[\Not \fE_2]{\ccE_1, \ccE_2},
      \ccE_2} \\
    &\equiv \chc[(\fE_1 \land \Not \fE_2)]{\ccE_1, \ccE_2} \qedhere
  \end{align*}
\end{proof}

The choice merge rules are stated in \figref{ccmerge}.
Unselectable alternatives in nested choices can be eliminated while preserving semantics by applying these rules from left-to-right.
Note that these rules are not independent of each other.
For example, we derive \rCCMerge\ from the other two choice merge rules in \thmref{ccmerge}.
Alternatively, we could derive \rCCMergeL\ and \rCCMergeR\ from \rCCMerge\ and \rCIdemp.

\begin{figure}[H]
  \onehalfspacing
  \begin{mathpar}
    \inferrule[\rCCMerge]
      {}
      { \chc{
          \chc{\ccE_1, \ccE_2},
          \chc{\ccE_3, \ccE_4}}
        \equiv
        \chc{\ccE_1, \ccE_4} } \\

    \inferrule[\rCCMergeL]
      {}
      { \chc{
          \chc{\ccE_1, \ccE_2},
          \ccE_3}
        \equiv
        \chc{\ccE_1, \ccE_3} }

    \inferrule[\rCCMergeR]
      {}
      { \chc{
          \ccE_1,
          \chc{\ccE_2, \ccE_3}}
        \equiv
        \chc{\ccE_1, \ccE_3} }
  \end{mathpar}
  \caption{Choice merge rules.}
  \label{fig:ccmerge}
\end{figure}

\begin{theorem}
  \label{thm:ccmerge}
  The choice merge rules hold for all formulas $\fE_1, \fE_2 \in \fS$, and expressions $\ccE_1, \ccE_2, \ccE_3, \ccE_4 \in \ccS$.
\end{theorem}

\begin{proof}
  First, we show \rCCMergeL\ directly from the definition of expression equivalence, see \appref{fcc-coq}.
  Next, we show \rCCMergeR\ by \rCCMergeL.
  The derivation of \rCCMergeR\ is as follows.
  \begin{align*}
    \chc{\ccE_1, \chc{\ccE_2, \ccE_3}} &\equiv
    \chc[\Not\fE]{
      \chc[\Not\fE]{\ccE_3, \ccE_2},
      \ccE_1} \\
    &\equiv \chc[\Not\fE]{\ccE_3, \ccE_1} \\
    &\equiv \chc{\ccE_1, \ccE_3}
  \end{align*}
  Finally, \rCCMerge\ follows directly from the other two rules.
\end{proof}

The object-choice commutation rule is stated in \figref{occomm}.
Although this rule refers to formula choices directly, it does not differ from the corresponding rule for dimension choices in a meaningful way.
For completeness, we include a proof of this rule in \appref{fcc-coq}.

\begin{figure}[H]
  \onehalfspacing
  \begin{mathpar}
    \inferrule[\rCOSwap]
      {}
      { \chc{
          \tr{\ccE_1, \ccE_2},
          \tr{\ccE_3, \ccE_4}}
        \equiv
        \tr{
          \chc{\ccE_1, \ccE_3},
          \chc{\ccE_2, \ccE_4}} }
  \end{mathpar}
  \caption{Object-choice commutation rule.}
  \label{fig:occomm}
\end{figure}

The choice-choice commutation rules are stated in \figref{ccswap}.
%Redundant alternatives in nested choices can be eliminated while preserving semantics by applying \rCCSwapL\ or \rCCSwapR\ from right-to-left.
Observe that we derived \rCCSwapL\ from \rCCSwap\ and \rCIdemp\ in the proof of \thmref{ccswap}.
Alternatively, we could derive \rCCSwap\ from the other two choice-choice commutation rules, and the choice merge rules.

\begin{figure}[H]
  \onehalfspacing
  \begin{mathpar}
    \inferrule[\rCCSwap]
      {}
      { \chc[\fE_1]{
          \chc[\fE_2]{\ccE_1, \ccE_2},
          \chc[\fE_2]{\ccE_3, \ccE_4}}
        \equiv
        \chc[\fE_2]{
          \chc[\fE_1]{\ccE_1, \ccE_3},
          \chc[\fE_1]{\ccE_2, \ccE_4}} } \\

    \inferrule[\rCCSwapL]
      {}
      { \chc[\fE_1]{
          \chc[\fE_2]{\ccE_1, \ccE_2},
          \ccE_3}
        \equiv
        \chc[\fE_2]{
          \chc[\fE_1]{\ccE_1, \ccE_3},
          \chc[\fE_1]{\ccE_2, \ccE_3}} }

    \inferrule[\rCCSwapR]
      {}
      { \chc[\fE_1]{
          \ccE_1,
          \chc[\fE_2]{\ccE_2, \ccE_3}}
        \equiv
        \chc[\fE_2]{
          \chc[\fE_1]{\ccE_1, \ccE_2},
          \chc[\fE_1]{\ccE_1, \ccE_3}} }
  \end{mathpar}
  \caption{Choice-choice commutation rules.}
  \label{fig:ccswap}
\end{figure}

\begin{theorem}
  \label{thm:ccswap}
  The choice-choice commutation rules hold for all formulas $\fE_1, \fE_2 \in \fS$, and expressions $\ccE_1, \ccE_2, \ccE_3, \ccE_4 \in \ccS$.
\end{theorem}

\begin{proof}
  First, we show \rCCSwap\ directly from the definition of expression equivalence, see \appref{fcc-coq}.
  Next, we show \rCCSwapL\ by \rCCSwap\ and \rCIdemp.
  The derivation of \rCCSwapL\ is as follows.
  \begin{align*}
    \chc[\fE_1]{
      \chc[\fE_2]{\ccE_1, \ccE_2},
      \ccE_3}
    &\equiv
    \chc[\fE_1]{
      \chc[\fE_2]{\ccE_1, \ccE_2},
      \chc[\fE_2]{\ccE_3, \ccE_3}} \\
    &\equiv
    \chc[\fE_2]{
      \chc[\fE_1]{\ccE_1, \ccE_3},
      \chc[\fE_1]{\ccE_2, \ccE_3}}
  \end{align*}
  Finally, we show \rCCSwapR\ by \rCCSwapL.
  The derivation of \rCCSwapR\ is as follows.
  \begin{align*}
    \chc[\fE_1]{
      \ccE_1,
      \chc[\fE_2]{\ccE_2, \ccE_3}}
    &\equiv
     \chc[\Not\fE_1]{
       \chc[\fE_2]{\ccE_2, \ccE_3},
       \ccE_1} \\
    &\equiv
    \chc[\fE_2]{
      \chc[\Not\fE_1]{\ccE_2, \ccE_1},
      \chc[\Not\fE_1]{\ccE_3, \ccE_1}} \\
    &\equiv
    \chc[\fE_2]{
      \chc[\fE_1]{\ccE_1, \ccE_2},
      \chc[\fE_1]{\ccE_1, \ccE_3}} \qedhere
  \end{align*}
\end{proof}

\section{Generalizations}
\label{sec:ccgen}

%It is difficult to extend formula choices with more alternatives.
Similar to dimension choices, formula choices can be extended to support more alternatives by adding tags to the set of tags and corresponding cases to the semantics of choices.
%In this way, the number of alternatives is equal to the number of tags.
However, the set of tags must also form a Boolean algebra---unlike the corresponding extension for dimension choices.
Since any non-trivial Boolean algebra has an even number of elements, it follows that formula choices cannot be extended with an odd number of alternatives in this way.

To see why any non-trivial Boolean algebra has an even number of elements, note that any Boolean algebra is a group with law of composition given by symmetric difference.
In such a group, every element is its own inverse.
This means the order of a non-identity element is two, which means the order of the group is a multiple of two by Lagrange's theorem~\citep[p.~89]{dummitfoote}.

Notwithstanding this limitation, formula choices can be extended with an even number of alternatives.
Consider the following example of extending formula choices with four alternatives.
The set $\tagS = \set{\tagLL, \tagLR, \tagRL, \tagRR}$ together with the operators defined by the tables in \figref{tag-ext} is the Boolean algebra with four elements---up to isomorphism---where \tagLL\ is ``true'' and \tagRR\ is ``false.''
The syntax and semantics of formulas is defined in the same way as before.
For expressions, we extend the syntax and semantics of choices in a similar way to the example extension from \secref{ccback}, albeit with four alternatives and cases instead of three.

\begin{figure}[H]
  \onehalfspacing
  $$
    \begin{array}{c|c}
      \tagE & \Not \tagE \\
      \hline
      \tagLL & \tagRR \\
      \tagLR & \tagRL \\
      \tagRL & \tagLR \\
      \tagRR & \tagLL \\
    \end{array}
    \qquad
    \begin{array}{c|cccc}
      \lor & \tagLL & \tagLR & \tagRL & \tagRR \\
      \hline
      \tagLL & \tagLL & \tagLL & \tagLL & \tagLL \\
      \tagLR & \tagLL & \tagLR & \tagLL & \tagLR \\
      \tagRL & \tagLL & \tagLL & \tagRL & \tagRL \\
      \tagRR & \tagLL & \tagLR & \tagRL & \tagRR \\
    \end{array}
    \qquad
    \begin{array}{c|cccc}
      \land & \tagLL & \tagLR & \tagRL & \tagRR \\
      \hline
      \tagLL & \tagLL & \tagLR & \tagRL & \tagRR \\
      \tagLR & \tagLR & \tagLR & \tagRR & \tagRR \\
      \tagRL & \tagRL & \tagRR & \tagRL & \tagRR \\
      \tagRR & \tagRR & \tagRR & \tagRR & \tagRR \\
    \end{array}
  $$
  \caption{Boolean algebra on four tags.}
  \label{fig:tag-ext}
\end{figure}

The equivalence rules for choices must be modified as well.
The choice merge rule \rCCMerge\ is replaced with the following rule:
\begin{mathpar}
  \inferrule%[\rCCMerge]
    { \ccE_i = \chc{\ccE_{i1}, \ccE_{i2}, \ccE_{i3}, \ccE_{i4}} \\
      \Forall i \in \set{1, 2, 3, 4} }
    { \chc{\ccE_1, \ccE_2, \ccE_3, \ccE_4}
      \equiv
      \chc{\ccE_{11}, \ccE_{22}, \ccE_{33}, \ccE_{44}} }
\end{mathpar}
Note that there are four premises of this rule.
The choice-choice commutation rule \rCCSwap\ is replaced with the following rule:
\begin{mathpar}
  \inferrule%[\rCCSwap]
    { \ccE_i = \chc[\fE']{\ccE_{i1}, \ccE_{i2}, \ccE_{i3}, \ccE_{i4}} \\
      \ccE_i' = \chc{\ccE_{1i}, \ccE_{2i}, \ccE_{3i}, \ccE_{4i}} \\
      \Forall i \in \set{1, 2, 3, 4} }
    { \chc{\ccE_1, \ccE_2, \ccE_3, \ccE_4}
      \equiv
      \chc[\fE']{\ccE_1', \ccE_2', \ccE_3', \ccE_4'} }
\end{mathpar}
Note that there are eight premises of this rule.
Similar modifications must be made to other rules.
